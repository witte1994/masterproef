\section{Usability test}\label{usabilitytest}

As the prototype neared completion, a usability test was designed. As the name implies, the purpose of such a test is to assess the usability of a prototype. While the prototype is being tested by a user, the researcher takes note of any usability issues that may arise. This evaluation method results in valuable feedback, as it shows directly how real users use a system~\cite{Nielsen1993}.

In this section we describe the design of a usability test to assess the user experience of the dashboard prototype. This test will focus on the usability aspect of the prototype, and not on the extensibility. However, in the future work section (\ref{future_work}) we give suggestions to test this aspect. All the documents created for the test are found in appendix~\ref{appendix_test_docs}.

    \subsection{Purpose}

    The goal of this test is to evaluate the usability of the dashboard prototype. The modules were designed with the usability principles mentioned in section~\ref{usability} in mind. Customization was added by giving the user more freedom in defining the functionality present in the dashboard. However, the experience of the end-users will determine if this is of added value when used in a clinical setting. Therefore, we are interested in gathering the following information:
    \begin{itemize}
        \item Is the dashboard easy to use? Is there a low or high risk of errors?
        \item Is the dashboard easy to navigate? Are all functions easily found?
        \item Does the dashboard as a whole succeed in displaying information in a clear and concise manner? How about the individual modules?
        \item Is the layout customization practical? Is it fast, slow, error-prone\ldots?
        \item How does the usability of the prototype compare to the system the tester uses?
    \end{itemize}

    \noindent Depending on the answers we receive to these questions, we can determine the next steps to take in future research, mentioned in section~\ref{future_work}. While our primary concern is the evaluation of our prototype, we also want to gain insight on the EHR system(s) the tester uses. This way, we can identify potential improvements or additional features for our prototype going forward. % chktex 36

    \subsection{Participants}\label{test_participants}

    The prototype is designed for use in different medical settings, which are staffed by many different roles. To get the best possible results, we want our group of participants to cover as many different roles as possible. For example, an general care nurse has completely different expectations from an EHR system compared to a paramedic. This will yield feedback we would otherwise not have received if all participants had the same role.

    Therefore, we need variety. Our participants should represent a wide spectrum of the health care industry in terms of the roles they fulfill. On top of that, having participants from different age groups is encouraged. Given the scope of this thesis, we chose the following sectors and roles to gather participants from:
    \begin{itemize}
        \item Laboratory unit: conducting tests and management of lab orders and results.
        \item Intensive care unit: requires close/real-time monitoring of several patients.
        \item General inpatient care: for example non-critical pediatric care within a hospital.
        \item Emergency response: for example a paramedic. Needs to respond fast, handle stressful situations, and needs to be mobile.
        \item A general practitioner: sees many patients daily with a large variety of conditions.
    \end{itemize}

    \noindent Now that we know who we want to test, the next question is how many. Nielsen determined that five participants find almost as much usability problems as for example a group of ten~\cite{Nielsen2012}. In other words, five participants result in the optimal benefit-cost ratio associated with user testing. However, there are exceptions to this rule. A quantitative study for example should test a lot more individuals in order to get statistically significant results.

    Our test is qualitative in nature, since we want to get a deeper understanding of what the needs of the end-user are. We are interested in \emph{why} the user experiences certain things. Therefore, according to Nielsen, we don't need to gather many participants, but preferably still more than five. This is due to the fact that the end-users of the dashboard have different professions and work in different care settings. The needs of each participant vary greatly between these settings, which is why five participants may fall short in uncovering some important usability issues.

    \subsection{Test structure}

    During the recruitment, willing participants may choose when and where the test will take place, given the location will not cause any disturbances. After meeting up, the following procedure takes place, with the estimated time for each step:
    \begin{myenumerate}
        \item Pre-test, 5 to 10 minutes:
        \begin{myenumerate}
            \item Greet and brief the participant.
            \item Ask for informed consent.
            \item The participant fills in a questionnaire which asks for demographic information and information concerning the EHR system he/she currently uses.
        \end{myenumerate}
        \item Test of the dashboard prototype, 20 to 25 minutes:
        \begin{myenumerate}
            \item Give the participant a brief tour of the prototype.
            \item The participant starts following the step-by-plan while being observed. No help or suggestions will be given unless necessary.
        \end{myenumerate}
        \item Post-test, 10 minutes:
        \begin{myenumerate}
            \item The participant fills in a second brief questionnaire about the user experience.
            \item The participant is interviewed. We go over suggestions, what was good/bad, future modules\ldots Additional questions may be asked as the interview goes on.
            \item Debrief the participant and hand out reward.
        \end{myenumerate}
    \end{myenumerate}

    \noindent The informed consent form found in appendix~\ref{appendix_test_consent}, is printed twice and filled in on the spot. The two questionnaires discussed in the next section, were created in Google Forms and are filled in using the laptop which has the prototype installed. Finally, the prototype is tested via the Google Chrome browser. A backup of the database is restored between each test to ensure that every participant will work with the exact same data.

    \subsection{Data gathering}

    The test results in quantitative data from two questionnaires. However, the qualitative information is more important, which is gathered from observing the participant during the test of the prototype and from the interview afterwards. Both questionnaires are found in appendix sections~\ref{appendix_pretest} and~\ref{appendix_posttest}.

    \paragraph{Quantitative data} The first questionnaire asks at the start for demographic information of which the following can be qualified a quantitative data: 
    \vspace{-6pt}
    \begin{myitemize}
        \item Age.
        \item Technology experience (Likert scale, 1: not experienced, 5: experienced).
    \end{myitemize}
    \noindent The second part asks for information regarding the current EHR system that the participant uses: 
    \vspace{-6pt}
    \begin{myitemize}
        \item Satisfaction (Likert scale, 1: very dissatisfied, 5: very satisfied).
        \item User friendliness (Likert scale, 1: strongly disagree, 5: strongly agree).
        \item Supports the participant's needs (Likert scale, 1: strongly disagree, 5: strongly agree).
    \end{myitemize}

    \noindent The second questionnaire asks for ratings regarding the usability of the prototype: 
    \vspace{-6pt}
    \begin{myitemize}
        \item General experience (Likert scale, 1: very bad, 7: very good). Because this is the general opinion of the participant, more steps are defined for this scale.
        \item Easy to use (Likert scale, 1: strongly disagree, 5: strongly agree).
        \item Dashboard looks clean (Likert scale, 1: strongly disagree, 5: strongly agree).
        \item Modules look clean (Likert scale, 1: strongly disagree, 5: strongly agree).
        \item Modules served as a good example (Likert scale, 1: strongly disagree, 5: strongly agree).
        \item Enough customization options present (Likert scale, 1: strongly disagree, 5: strongly agree).
    \end{myitemize}

    \noindent No other quantitative data is gathered. During the observation, errors are noted, but not counted.

    \paragraph{Qualitative data} The bulk of the qualitative data is gathered during the test of the prototype by observation and afterwards during the interview. When the participant is following the step-by-step plan, the following notes may be taken:
    \vspace{-6pt}
    \begin{myitemize}
        \item The participant struggled finding UI element X.
        \item An error was made regarding X.
        \item The participant hesitated at step X.
        \item The participant asked question X regarding Y\ldots
    \end{myitemize}

    \noindent Questions related to the usability of the prototype and the current EHR system, current paper use in their care setting, module suggestions\ldots are asked during the interview. Section~\ref{discussion} describes what qualitative data was retrieved.

    \subsection{Test procedure}

    During the step-by-step process, the participant works with the dashboard of three fictional patients. A short background is given for each patient and participant performs actions which relate to the situation of the patient. The participant has seen all aspects of the dashboard after the test of the prototype is complete. Appendix~\ref{appendix_test_steps} contains the step-by-step process. Also, the screenshots in appendix~\ref{appendix_test_screens} show the dashboard at the start of each scenario. We now describe each scenario.

    \paragraph{Patient 1} Background: Kenny is an account manager of a large accounting firm. The combination of his stressful job, sedentary lifestyle, and smoking habit have lead to health issues. He has frequent palpitations and high blood pressure. At home, Kenny has to measure his weight, heart rate, and blood pressure on a regular basis with devices that send the values to his electronic patient record. In this scenario the participant does the following:
    \vspace{-6pt}
    \begin{myenumerate}
        \item Apply a filter on the history module.
        \item Add and configure a telemonitoring module to fill the empty space of the dashboard.
        \item Reorder the modules on the summarizing panel, while adding and configuring a small telemonitoring module.
    \end{myenumerate}

    \paragraph{Patient 2} Background: Since Jozefien retired, she has been gaining weight at an alarming rate. As a countermeasure, she regularly visits her general practitioner. Each consultation, the practitioner updates her medication scheme and tells her what she needs to pay attention to. In this scenario the participant does the following:
    \vspace{-6pt}
    \begin{myenumerate}
        \item Remove and add a prescription, check for interactions.
        \item Remove the allergy module to replace it with a workflow module.
        \item Edit some steps of a workflow.
        \item Reset the checklist and add some new tasks to it.
        \item Hide a parameter in the small telemonitoring module found in the left panel.
    \end{myenumerate}

    \paragraph{Patient 3} Background: After a serious car accident, Bert is hospitalized in the intensive care unit. Bert has many allergies, an elaborate medication scheme, and misses some vaccinations. Currently he is being treated by several clinicians and being observed by nurses. It is important that everyone is aware of these allergies, the medication scheme, and missing vaccinations. However, the data in the EHR system is not up to date. The latest most up to date medical data is still stored in a paper dossier. In this scenario the participant does the following:
    \vspace{-16pt}
    \begin{myenumerate}
        \item Change and add an allergy.
        \item Remove and add a vaccination.
        \item Apply a filter on the history module.
    \end{myenumerate}