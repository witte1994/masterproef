\section{Discussion}\label{discussion}

The search for participants began immediately after the usability test design was completed. This section describes the recruitment process. Hereafter the findings of the usability tests are noted.

    \subsection{Recruiting participants}

    Recruitment started on the 11th of December, 2018. A total of 13 people were contacted with the following backgrounds:
    \vspace{-6pt}
    \begin{myitemize}
        \item 2 nursing students, both had internship experience in a hospital setting.
        \item 1 elderly care nurse.
        \item 2 intensive care unit nurses.
        \item 1 clinical laboratory department head.
        \item 1 paramedic.
        \item 2 home nurses, working for the same practice.
        \item 4 general practitioners, all work together under one practice.
    \end{myitemize}

    \noindent Five individuals immediately were scheduled to test the prototype. The two home nurses and one nursing student declined due to time constraints. The paramedic expressed interest but was unreachable for two weeks until the start of the new year. The practice with the four general practitioners were contacted three times via their shared secretary, twice via phone and one time in person. The secretary promised each time to provide an answer the next day via email, but this did not happen. Six participants have tested the prototype. The first five tests were conducted between the 16th and 26th of December, and the last one on the 5th of January.

    \subsection{Results}

    All conducted tests faced no issues regarding the application itself. Every participant tested the prototype in a secluded and silent environment. Four tests were conducted at the homes of the participants and the other two in empty classrooms of Hasselt University. The test was estimated to take approximately 40 minutes to complete. In anticipation of lengthy interviews, a hard stop was put in place should the test pass the 60 minute mark. This was the case for two tests. The durations of the other four were between 45 and 55 minutes. This caused no issues for any of the participants. When the test concluded, participants were rewarded with a bottle of white Sicilian wine with a price tag of \euro 4,29. During the recruitment phase, each contacted individual was aware of a reward, but they did not know what it was beforehand. % chktex 1

    \subsubsection{Pre-test questionnaire}

    Table~\ref{table:pre-test} shows some general information from the pre-test questionnaire regarding the participants. It should be noted that 5 out of 6 participants are between the ages 21 and 25. They also indicate a higher familiarity with technology, compared to the older participant. The younger participants also have limited work experience. The pre-test questionnaire also indicated that all participants use their smartphone and lap-/desktop daily. Four participants use their tablet a few times every month, of which one participant uses it daily. One participant uses a smartwatch daily, and is the only one to use a smartwatch at all.
    
    \begin{table}[!t]
        \resizebox{\textwidth}{!}{%
        \begin{tabular}{ccllc}
            \hline % chktex 44
            \textbf{Age} & \textbf{Gender} & \textbf{Profession}        & \textbf{Experience} & \textbf{\begin{tabular}[c]{@{}c@{}}Tech\\ experience (1-5)\end{tabular}} \\ \hline % chktex 44
            21           & M               & Nursing student            & 4 years internship  & 4                                                                  \\
            25           & F               & Elderly care nurse         & 1,5 years           & 4                                                                  \\
            23           & F               & Intensive care unit nurse  & 1 year 5 months     & 4                                                                  \\
            23           & F               & Intensive care unit nurse  & 5 months            & 4                                                                  \\
            49           & F               & Laboratory department head & 28 years            & 3                                                                  \\
            22           & M               & Paramedic                  & 1 year              & 5
        \end{tabular}%
        }
        \caption{Pre-test results: general participant information}\label{table:pre-test}
    \end{table}

    Table~\ref{table:pre-test-ehr} shows that all participants use a different EHR systems for their care setting and only one of them is used on a mobile device. The user satisfaction of the EHR systems scored an average of 4 out 5, with user friendliness scoring a bit less. The participants were more neutral towards the features the EHR system provides, scoring a 3,17 on average. Only two EHR systems provide customization. Lastly, one participant uses no less than 6 applications in combination with the EHR system, while two participants use none. During the interview, more questions were asked concerning their current EHR system.

    \begin{table}[!t]
        \resizebox{\textwidth}{!}{%
        \begin{tabular}{lcccccl}
        \hline
        \textbf{EHR system} & \textbf{Mobile} & \textbf{\begin{tabular}[c]{@{}c@{}}Satisfaction\\ (1-5)\end{tabular}} & \textbf{\begin{tabular}[c]{@{}c@{}}User friendliness\\ (1-5)\end{tabular}} & \textbf{\begin{tabular}[c]{@{}c@{}}Complete\\ toolset (1-5)\end{tabular}} & \textbf{\begin{tabular}[c]{@{}c@{}}Custom-\\ izable\end{tabular}} & \multicolumn{1}{c}{\textbf{\begin{tabular}[c]{@{}c@{}}\# other \\ systems\end{tabular}}} \\ \hline
        Orbis               & No              & 3                                                                     & 3                                                                          & 2                                                                         & No                                                                & 0                                                                                        \\
        GEMS                & No              & 3                                                                     & 3                                                                          & 3                                                                         & No                                                                & 6                                                                                        \\
        Metavision \& GEMS  & No              & 5                                                                     & 4                                                                          & 4                                                                         & No                                                                & 2+                                                                                       \\
        ICCA                & No              & 5                                                                     & 5                                                                          & 4                                                                         & Yes                                                               & 4                                                                                        \\
        HIX                 & No              & 4                                                                     & 4                                                                          & 3                                                                         & Yes                                                               & 1                                                                                        \\
        KWS                 & Yes             & 4                                                                     & 3                                                                          & 3                                                                         & No                                                                & 0
        \end{tabular}%
        }
        \caption{Pre-test results: EHR systems in use by participants. Note: same order is respected as table~\ref{table:pre-test}.}\label{table:pre-test-ehr}
    \end{table}

    \subsubsection{Prototype test: observations}

    All participants were able to complete the test within the allotted time of 20 to 25 minutes. The two intensive care nurses were noticeably faster in completing all the steps, nearing the 15 minute mark. The elderly care nurse and the laboratory head were often unsure on where to click and often asked questions before trying. This may explain the fact that they took a bit longer to complete the test compared to the other participants. However, every participant received the same brief tour of the prototype, so no participant had more knowledge of the prototype compared to the others. We now go over the comments made by the participants during the test.

    There were several comments regarding the prescription module. One participant noted that it would be useful if empty dosage fields were automatically filled with zeroes. Currently, the user needs to fill all fields manually. Another participant suggested something similar. Because today's date had to be filled in multiple times throughout the test, the participant would've liked a button that would automatically fill it in. This participant stressed that error checking was a very important component of the systems they currently use. Regarding the end date of prescriptions, one participant noted that they sometimes have to administer medication indefinitely, which the module does not support at the moment. In case there are interactions between medicines, one participant noted that it would be helpful to see the actual interaction effects in addition to the interacting medicine.

    The test highlighted two issues. First, four participants struggled to add a new task to the checklist. This was done by opening the context menu on top of the checklist description. In this case, a button would be a better option. Second, three participants had difficulties finding the option to add a vaccination entry, which again was found in a context menu. More careful thought must be given on when to use these context menus. However, one participant had no issues with the context menu and thought it was really useful. This participant shared this thought also for the date picker.

    Participants had a few more comments. Instead of clicking the arrow button to open the dashboard of a patient, the user can click anywhere in the row of the table to open it. Also, one participant liked both the normal and small telemonitoring modules in particular, describing them as very clean and simple. To conclude, a participant described the importance of access control and privacy surrounding the patient data history module. 
    
    \subsubsection{Post-test questionnaire}

    Table~\ref{table:post-test} shows the results of the Likert scale questions concerning the usability of the prototype. The results are very positive, indication good overall experience with the dashboard. Furthermore, all participants indicated that they had no trouble finding their way around the dashboard. However, these results are nowhere near conclusive, due to the small sample size. Also, the fifth participant was the only older person in the sample group. The general experience of the dashboard and the experience with technology were both lower for this participant. But then again, a laboratory setting does not share many similarities with the other settings found in the sample group. 
    
    Maybe the most important result is that the modules served as good examples of what is possible with the dashboard. This allowed the participants to better relate the prototype to their work setting, which may lead to valuable feedback during the interview. Also, the questionnaire asked the participants to suggest some modules which would fit the dashboard, which we discuss in more detail in the next section.

    \begin{table}[!t]
        \resizebox{\textwidth}{!}{%
        \begin{tabular}{cccccc}
        \hline
        \textbf{\begin{tabular}[c]{@{}c@{}}General\\ experience\\ (1-7)\end{tabular}} & \textbf{\begin{tabular}[c]{@{}c@{}}Ease\\ of use\\ (1-5)\end{tabular}} & \textbf{\begin{tabular}[c]{@{}c@{}}Dashboard\\ clear UI\\ (1-5)\end{tabular}} & \textbf{\begin{tabular}[c]{@{}c@{}}Modules\\ clear UI\\ (1-5)\end{tabular}} & \multicolumn{1}{l}{\textbf{\begin{tabular}[c]{@{}l@{}}Modules good\\ examples (1-5)\end{tabular}}} & \textbf{\begin{tabular}[c]{@{}c@{}}Enough\\ Customization\\ (1-5)\end{tabular}} \\ \hline
        6                                                                             & 5                                                                      & 5                                                                             & 5                                                                           & 4                                                                                                  & 5                                                                               \\
        6                                                                             & 4                                                                      & 3                                                                             & 4                                                                           & 4                                                                                                  & 4                                                                               \\
        6                                                                             & 4                                                                      & 5                                                                             & 4                                                                           & 5                                                                                                  & 5                                                                               \\
        7                                                                             & 5                                                                      & 5                                                                             & 5                                                                           & 5                                                                                                  & 5                                                                               \\
        4                                                                             & 3                                                                      & 4                                                                             & 4                                                                           & 4                                                                                                  & 4                                                                               \\
        7                                                                             & 5                                                                      & 5                                                                             & 5                                                                           & 5                                                                                                  & 5                                                                               \\ \hline
        \textbf{6}                                                                    & \textbf{4,33}                                                          & \textbf{4,5}                                                                  & \textbf{4,5}                                                                & \textbf{4,5}                                                                                       & \textbf{4,67}                                                                   \\ \hline
        \end{tabular}%
        }
        \caption{Post-test results: questionnaire to assess usability of the dashboard. The bottom row indicates the average scores. Note: same order is respected as table~\ref{table:pre-test}.}\label{table:post-test}
    \end{table}

    \subsubsection{Interview}

    Several topics were brought up during the interview, such as the current EHR system that is in use, the existence of paper in the work setting, potential modules to add to the dashboard, use cases for the current modules, what was good or bad, and other suggestions.