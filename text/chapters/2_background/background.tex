% definition, purpose
\section{The Electronic Health Record}\label{background}

Health information systems have become an integral part of health care. They support patient care as well as administrative and financial tools. At the heart of these systems lies the electronic health record. An electronic health record (EHR) is a repository of electronically maintained information about an individual's health status and health care, stored such that it can serve multiple legitimate uses and users of the record\cite{biomedical_informatics}. An electronic health record system (EHRS) provides tools to manage and interact with these records. These tools include reminder generation, data analysis, and decision support. It helps the clinician to organize, interpret and react to medical data. A record where the patient manages its health information is called a personal health record\cite{tang2006personal}.

The first section describes the transition from paper-based records to digital and its implications for the medical world. A summary of the main components present in an EHRS is described in the next section. The functionality of an EHRS can be categorized into two types: a monolithic system tries to provide more general care, while smaller and more specialized systems cater to more specific types of care. The last section compares the two types and highlights the benefits and drawbacks of both.

No existing EHRS were reviewed. Access to these systems is limited to open-source solutions as most are commercial and sometimes off-the-shelf products. Consequently, reviews of these systems are difficult to find as institutions only purchase software after thorough review of its features. Only one article was found which compared three open-source health systems\cite{de2012overview}. However, no literature was found for proprietary systems.

    % nog altijd veel in gebruik
    % signify importance
    \subsection{Moving away from paper}\label{2_ehrs_paper}

    For modern medicine, traditional paper-based medical records are not suited for today's world filled with technology. The drawbacks of information on paper are obvious when compared to digitally stored information.

    \paragraph{Functionality} Digital records allow systems to aggregate, process and create statistics of the data it contains. For example, a system can generate heart rate graphs with statistics, summarizing many values. If the user hovers over a data point in the graph, the exact value is shown. Printed tables and graphs showing the same data lack this interaction. Also, paper records demand more effort from clinicians as data is often spread across multiple dossiers.

    A paper-based medical dossier can store for example medical images, such as x-rays. Compared to a digital image, paper loses a significant amount of detail. As such, most multimedia types can only be stored digitally and not on paper. Tools to interact with these file types are provided by an EHRS\@.

    In terms of data input, an EHRS can detect and prevent false data input. The system ensures that all necessary data fields are filled in and in the correct format. This results in more complete and accurate data gathering with fewer errors, increasing the information quality.

    \paragraph{Information quality} A summarizing paper noted that the use of EHRS leads to more complete, accurate, comprehensive, and reliable data compared to paper-based records\cite{ehrs_summary}. As mentioned in the previous paragraph, a digital system can impose rules on data fields to avoid missing or entering the wrong data. In terms of comprehensibility, poor handwriting leads to wrong or loss of information, which digital systems avoid. Medical instruments can save measurements immediately into system, avoiding copying by hand.

    \paragraph{Accessibility} Paper records are difficult to access because most of the time only one copy exists. Therefore, transferring these records to other branches or institutions is difficult as the record needs to be located in the often large medical dossier. Also, misfiling, flooding or fires lead to irrecoverable loss of the data. Creating backups of the digital records avoids the last issue, but difficulties surrounding data transfer are still present. Most institutions have their own database structures which hinders the transfer of raw medical data. To remedy this, interfaces are created by the IT staff to interpret and store the data. Sending for example PDF-reports via email is a convenient alternative, although limited in functionality comparable to an actual paper record. Therefore, integration capabilities and the use of standards (section \ref{2_privacy}) are important for an EHRS\@. % chktex 2

    Moving data from paper to a digital format requires a lot of manual work. While automation is possible, such as scanning and processing the paper forms with software, verification is still necessary. All the data fields from the documents need to find a place in the system, which is difficult to achieve. Extra diagnostic information scribbled all over the the document, will be lost during this process. Also, what do we do with unreadable data due to poor handwriting? What happens with two forms which contain partly the same information? Do we save the information twice or do we add extra checks to prevent duplication? Because of these reasons, adopting an EHRS is a significant undertaking for an institution. As such, the benefits are not immediately apparent.

    Because digital storage increases accessibility significantly, security measurements have to be taken. If not, data can be stolen, deleted or even altered, which means a breach of privacy (section \ref{2_privacy}). This also adds to the complexity of developing EHRS\@. % chktex 2

    \paragraph{Efficiency} An important task of an EHRS is to facilitate effective care. An early study saw a 6\% increase in productivity when EHRS were deployed in health care institutions\cite{ehrs_efficiency}. However, other factors are at play, such as the adoption rate. The transition from paper-based documenting to digital is accompanied by a learning curve, which can be steep. After the switch, the productivity will most likely be lower at the start. As the users become more experienced with the software, it will increase. Today most institutions have already made the transition to an EHRS, so this has become less of an issue.

    % component
    \subsection{Components}\label{ehrs_components}

    As mentioned before, EHRS do not simply store patient records. They consist of many components which ultimately defines how well they perform in health care. Literature defines many lists of essential components. However, we define the following five high level components as key\cite{biomedical_informatics}:

    \paragraph{Integrated view of patient data} An EHR must allow storage of a wide range of data types. This can be text, numbers, images, video, and others. Some data can still be on paper due to lacking support of the EHRS, as mentioned in section \ref{2_ehrs_paper}. To display more complex data types, such as x-ray images, standards are used. A brief overview of medical standards is described in section \ref{2_standards}. % chktex 2

    \paragraph{Clinician order entry} The point at which the clinician enters treatment instructions is called order entry. An order entry system assists the clinician during the decision-making process to ensure that the instructions are correct. It also reduces errors and costs compared to paper order entry for the same reasons already mentioned in section \ref{2_ehrs_paper}. % chktex 2

    \paragraph{Clinical decision support} A decision support system embed into an EHRS aids the clinician by suggesting certain actions when certain situations occur. If for example, a patient is due for vaccination, the system notifies the clinician by presenting a constructed order which needs to be confirmed or denied. The system can do this for a bulk of patients, so manual checkups are not required, thus saving time.

    \paragraph{Access to knowledge resources} During the writing of notes or orders for a patient, clinical questions often arise. Instead of asking colleagues or searching through multiple manuals, the EHRS searches for relevant literature to address the question. Due to the internet, a very large source of information is readily available.

    \paragraph{Integrated communication and reporting support} Communication is an important part of health care. Often clinicians spread across multiple institutions provide care for the same patient. Communication, therefore, directly affects the effectiveness of patient care. Tools that simplify this process are essential for a health system. 

    Most institutions are bound by their own EHRS\@. If data resides in another institution's health system, then access has to be requested. Health Information Exchange removes the need to manually ask for data access as institutions are able to reach beyond their own system. This, however, needs to be supported by the institution.\bigskip
    
    \noindent In section FILL, more specific components are listed, such as prescription management. Throughout the years, many EHRS have been developed which may or may not integrate all of the above components. Should an essential component be missing, an institution may add another software system. This has its own benefits and drawbacks.

    \subsection{Monolithic vs.\ multiple EHRS}\label{ehrs_comparison}

    Health care institutions can opt for a monolithic EHRS or combine multiple EHRS to achieve all required functionality. There advantages and drawbacks to both\cite{multiple_ehrs}. Elements that influence this choice include IT infrastructure, safety risks, the volume of care, and frequency with which patients move facilities. It is possible that a single EHRS does not satisfy the requirements of an institution. A reason for this is that certain branches require specially tailored software for their clinical practice, which the current EHRS in use lacks.

    Functionality wise, a monolithic EHRS tends to appeal to more general types of care, whereas it lacks in very specific ones. Also, vendors that offer these all-in-one solutions tend to have less experience with these special types of care which reduces the chance it will be added to the system. Vendors of specific EHRS do have this expertise and can tailor the system to the needs of the customer. In this case, combining a system that supports general care with special care systems seems like the best choice. However, other factors have to be considered.

    The advantage of a monolithic system is that the data it uses is centralized. This ensures that all data is accessible anywhere throughout the system and is easier to maintain. When multiple systems are in place, data has to be exchanged between them. As a result, searching for data is more difficult. As mentioned in section\ref{2_ehrs_paper}, due to different data structures, data exchange is difficult. To solve this issue, an intermediate EHRS can be developed. This system serves solely for the purpose of data collection and transformation. All other systems search for data in this intermediate system. However, this leads to another system, requiring additional costs, development effort, and maintenance.

    \subsection{Extensibility}

    \subsection{Customization}

    \subsection{Privacy}\label{2_privacy}

    Privacy refers to the desire of a person to control the disclosure of personal health and other information\cite{biomedical_informatics}. On the other hand, confidentiality is the ability to control the release of personal health information to a care provider under the agreement that the information will not be spread or used further. Security is the protection of privacy and security, which is achieved through policies, procedures, and safeguards.

    We can ask several questions related to health information data access and storage: who owns the data? Is it the health provider or the patient? What type and how much of data needs to be stored? Who can read the data? Who can write to the data? Can someone access specific health information without consent? In order to deal with the some of the challenges concerning data access and storage, these questions need to be answered\cite{meingast2006security}.

    An important issue surrounding privacy is medical data access. For example, researching the medical data of a large population can uncover looming threats or an epidemic. Also, disease incidence and intervention analysis benefits from data pooling. However, medical data should not be too accessible. As more and more people gain access to health data of the population, the chance that data is misused or confidentiality is broken increases. Therefore, striking a balance between free information access and protection of privacy and confidentiality is difficult. An argument that promotes free access is to anonymize the data by removing identifiers such as the person's name. However, the individual pieces of information can still reveal the their identity.

    Compared to paper records, electronic records are easily accessible, such as through the internet. This calls for extra security measures to prevent data breaches. If someone wants to access a medical record, this person must first authenticate himself. In turn, the system checks if the person is allowed to access this information. Regardless of the outcome, the audit system logs the access attempt. Such systems need clear rules and policies defined by the institution on who can access what. Another obvious security measure is data encryption.

    To further prevent malicious use of health data, medical staff should follow education and training programs to make them aware of the privacy rules and policies that are in place. In case of privacy violation, the person in question should be punished appropriately.

    \subsection{Standards}\label{2_standards}

    When excessive diversity creates inefficiencies and affects effectiveness standards are required\cite{biomedical_informatics}. A hospital contains many independent units spread across primary, secondary, and tertiary care. These units use software best fit for their practice and all record different types of data. For example, the hospital admissions system records patient diagnosis, the pharmacy records prescriptions that were handed out, and the laboratory system records test results. Inevitably, transfer of data between these units is required. 

    To coordinate multiple systems, data transfer is essential. Nowadays too many different systems exist to create point-to-point interfaces for. Standards try to resolve this problem by defining guidelines software system should follow in order to communicate data. An efficient standard requires that data can be easily stored and presented towards the users of an EHRS\@. Also, security measures, such as authentication and access control, need to be interwoven with these standards.

    The use of standards in an EHRS leads to easier integration which in turn leads to lower development costs. Integration of many proprietary data structures leads to difficult to maintain software and a higher chance of components breaking. New medical devices and software that comply to these standards prevent this.

    One of the most used and implemented standard set is Health Level 7. These messaging-based standards are created to exchange clinical and administrative data. Another standard such as DICOM is used for the communication and management of medical imaging information\cite{mildenberger2002introduction}. Medical device interface standards are created by IEEE\@. Many more standards exist, but these are the more prominent ones.