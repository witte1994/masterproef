% definition, purpose
\section{The Electronic Health Record}\label{background}

Health information systems have become an integral part of health care. They support patient care as well as administrative and financial tools. At the heart of these systems lies the electronic health record. An electronic health record (EHR) is a repository of electronically maintained information about an individual's health status and health care, stored such that it can serve multiple legitimate uses and users of the record\cite{Shortliffe2014}. An electronic health record system (EHRS) provides tools to manage and interact with these records. These tools include reminder generation, data analysis, order entry, and decision support. It helps the clinician to organize, interpret and react to medical data. A record where the  health information is managed by the patient, is called a personal health record\cite{Tang2006}.

    % nog altijd veel in gebruik
    % signify importance
    \subsection{Moving away from paper}\label{paper}

    For modern medicine, traditional paper-based medical records are not suited for today's world filled with technology. The drawbacks of information on paper are obvious when compared to digitally stored information.

    % misplaced, lost -> retests, duplication
    % difficult to organize
    % physical space
    % high costs, not eco friendly
    \paragraph{Storage} Paper records need to be stored in a safe location and require a lot of physical space. More so, the organization of these records is a daunting task in case they are fragmented across multiple locations. This is a process that wastes time and therefore increases health care costs. Also, losing or misplacing records is a possibility. If this for example happens to lab results, tests have to be redone which again wastes time and increases costs. Finally, the amount of paper added to the records on a weekly basis is significant and very eco-unfriendly\cite{Saleem2009}. The ink and paper costs should not be underestimated. Storing data in a digital format solves these issues, but raises several other questions such as: shall we store the data on premise or choose a cloud provider? What is our backup policy? 

    % not computable
    % incomplete -> poor quality care
    % illegible
    \paragraph{Data quality} The medical information available on paper is static and more effort is required from clinicians in processing it. For example, comparing data values by reading two forms side by side. What if the data is illegible due to poor handwriting or worn down over time? What if data is incorrectly read from the computer display or simply not written down? These medical errors directly impact the quality of care that is delivered\cite{Elnahal2011}\cite{Hillestad2005}. Computation is possible for digital records which allows data aggregation, data processing and statistics generation. Also, a computerized system detects false data input and ensures that all data fields are filled in correctly. A summarizing paper noted that the use of EHRS leads to more complete, accurate, comprehensive, and reliable data compared to paper-based records\cite{Hayrinen2008}.

    Medical data such as printed x-ray images are prone to data loss, which in turn increases the chance of medical errors. A high resolution monitor allows zooming, whereas the paper is severely limited in this regard. In case of audio files, bundling for example a cassette tape with a paper dossier is problematic in terms of storage and transportation. The last two examples can both be stored digitally and paired to a digital record.

    % sent via mail/scanned in or converted to digital lot of work and resources
    % 1 person at a time access to medical dossier, changes slow to reach other locations
    % limited by location
    \paragraph{Accessibility} Paper records are bound by their location and can often only be accessed by one person at a time. Transferring the paper record requires a lot of manual work: either the clinician sends the actual record or a copy of it, or converts the record to a digital format. One study described that updating several copies of medical records was a very cumbersome process for clinicians\cite{Tange1999}. If such a copy is updated, how can these changes be reflected in the other existing copies? Where are these copies even located? Converting the paper records to a digital format raises several other questions: where can the data values found in the paper record be stored in the digital system? What if there are extra notes scribbled on the document? What if values are missing? Data transfer and dealing with changes seems simpler for digital records, however the interoperability of different systems is complex. This is discussed in section \ref{interoperability}. % chktex 2

    % break-in: steal/alteration/destroy data
    % natural disaster
    % backups not feasible
    \paragraph{Security} While paper records can be safely stored under lock and key in a basement, several security issues still exist. For example, the storage location can be broken into. This allows the perpetrator to retrieve, alter, or destroy critical medical information. Other possibilities of data loss include flooding and fires. These are scenarios that can't be predicted and can happen at any time. 

    A common strategy to detect data alteration and prevent data destruction is to create regular backups. However, for paper-based medical records this is simply not feasible. As mentioned before, the resources required to create copies of such a large data set is enormous. For digital records this is a much easier process, but in this case other security issues arise. These are discussed in section FILl. Also, data breaches have important privacy implications. This topic is discussed in section \ref{privacy}.\bigskip % chktex 2

    \noindent Electronic health record systems deal with most of the issues mentioned in the previous paragraphs. However, there are other factors which are responsible for the continuing presence of paper in clinical environments. Three categories are predicted to be the cause of paper generation: policy requirements, suboptimal system design, and EHRS user interface flaws\cite{Saleem2009}. The study focused on the last source with the following subcategories:
    \begin{itemize}
        \item Cumbersome interface design leads to handwritten notes on paper.
        \item The EHRS is not well integrated into the clinical workflow. This leads to paper-based workarounds which \emph{do} align with the workflow.
        \item The visual organization of the data in the EHRS is incompatible with the mental model of the clinician, which leads to manual transformation of patient data.
    \end{itemize}

    \noindent Notice that these subcategories relate to usability and human-computer interaction issues. Twenty persons practicing varying roles in a clinical environment were interviewed concerning the use of paper in their workflow. Afterwards, recurrent paper-based workaround strategies highlighted during analysis were categorized. Eleven distinct categories were found, ordered by frequency of use:
    \begin{itemize}
        \item Efficiency: paper use enhanced perceived or actual efficiency.
        \item Workarounds related to the clinician's knowledge of the health system, skill level with technology, and ease of use of the health system. Example: using paper because the software is difficult to use.
        \item Memory: cases where paper is used as a reminder tool.
        \item Sensorimotor preferences: using paper as a means of having something concrete to deliver or to quickly jot down some notes.
        \item Awareness: using paper to make clinicians aware of new information.
        \item Task specificity: cases where the health system lacks specificity, is not customizable, or sends too many alerts (alert overload).
        \item Task complexity: paper processes are used because the health system does not support it. An oncology order is an example of a complex task, tailored specifically for each patient.
        \item Data organization: using paper when data is poorly organized on the computer screen.
        \item Longitudinal data processes: using paper instead of the health system when data needed to be tracked over time.
        \item Trust: using paper as a form of providing proof the health system cannot provide.
        \item Security: using paper when the health system is insecure.
    \end{itemize}

    \noindent While these workarounds can improve efficiency, they also circumvent the health system which minimizes medical errors frequently made when dealing with paper records. However, they also imply that the EHRS is not in line with the workflow of the clinicians. Several of these workarounds are caused by poor usability, which is the topic of section \ref{usability}. % chktex 2

    % component
    \subsection{Components}\label{components}

    As mentioned before, EHRS do not simply store patient records. They consist of many components which ultimately define how well they perform in health care. We define the following five functional components as key\cite{Shortliffe2014}:

    \paragraph{Integrated view of patient data} An EHR must allow storage of a wide range of data types. This can be text, numbers, images, video, and others. Some may still be on paper due to lacking support of the EHRS, as described in section \ref{paper}. Data standards were developed to store simple and complex data types, ranging from numeric values to x-ray images. Medical data standards are the central topic of section \ref{standards}. % chktex 2

    \paragraph{Clinician order entry} The procedure in which the clinician enters treatment instructions is called order entry. An order entry system assists the clinician during the decision-making process to ensure that the instructions are correct. It reduces medical errors as a result of more complete and correct data as mentioned before in section \ref{paper}. Also, costs associated with the use of paper are saved. % chktex 2

    \paragraph{Clinical decision support} A decision support system aids the clinician by suggesting actions when certain situations occur. If for example, a patient is due for vaccination, the system notifies the clinician by presenting a pre-filled order which only needs to be confirmed or denied. The system can do this for a bulk of patients, so manual checkups are not required, thus saving time. Decision-support implementations benefit from artificial intelligence. As such, these systems are often complex, due to the many parameters involved in the decision-making process.

    \paragraph{Access to knowledge resources} Clinical questions often arise during the clinician's workflow. Instead of asking colleagues or searching through multiple manuals, the clinician can consult the EHRS for literature. Thanks to the internet, large sources of information are readily available. Also, if the EHRS is aware of the current medical context, searching for the required information takes even less time. 

    \paragraph{Integrated communication and reporting support} Communication lies at the heart of health care delivery. It is common for patients to receive care from several clinicians spread across multiple institutions or departments. Noted in section \ref{paper}, using paper as a form of communication should be avoided. Therefore, the availability of communication tools directly affects the quality of care. The use of standards and high interoperability of health systems facilitate efficient communication (section \ref{interoperability}).\bigskip % chktex 2

    \noindent All components should be present in health software. This example of a prescription management tool, features all five components:
    \begin{itemize}
        \item Present list of medication currently taken in a clear manner.
        \item Create, edit, or remove prescriptions.
        \item Provide alerts for drug-drug and drug-allergy interactions. Suggest medication dosage with respect to parameters such as weight.
        \item Provide extra information regarding medicines, such as side effects.
        \item Send prescription orders to the pharmacy.
    \end{itemize}

    \noindent An EHRS features many tools such as prescription management and many examples are given in section \ref{module_brainstorm}. Ideally, all tools are bundled in a single software package. In case this is not possible, multiple applications are used to provide all the required functionality. As soon as applications need to communicate with each other, interoperability becomes very important, which is discussed in section \ref{interoperability}. % chktex 2

    \subsection{Impact}\label{impact_ehrs}

    Transforming the health care industry to use technology has a profound impact on many aspects, such as finances and efficiency. As mentioned in section \ref{paper}, paper is not reliable nor durable to provide efficient care. The identification and research process of all the potential benefits and drawbacks of EHRS is not simple. There are countless of practices, small and large, providing many types of care. As a result, most studies focus on one care setting at a time, such as general care provided by a hospital. Therefore, the effects of an EHRS found in one type of care setting should not be generalized to other settings. % chktex 2

    One study conducted in 2011 summarized the literature that studied the benefits and drawbacks of EHRS\cite{Menachemi2011}. As such, this study served as the primary source of information in this section. First, we describe the advantages of EHRS on several outcomes. Hereafter, we go over the disadvantages.

        \subsubsection{Potential advantages}\label{ehrs_advantages}

        The positive impact of EHRS was studied for clinical, organizational, and societal outcomes. For each of the outcomes a definition is given, together with the observed benefits.

        \paragraph{Clinical outcomes} Measurable changes observed in quality of care are related to this outcome. Quality of care can be defined as ``doing the right thing, at the right time, in the right way, for the right person, and having the best possible results''\cite{AHRQ2001}. Quality of care includes six dimensions\cite{CTQC2001}. However, studies focused mainly on the following three: effectiveness, efficiency, and patient safety.

        EHRS lead to increased adherence to evidence-based clinical guidelines, resulting in more effective care. Clinicians don't follow these guidelines because they don't know them, don't know it applies to the patient, or have insufficient time. The EHRS helps overcoming these issues. Computerized alerts are also linked to improving the effectiveness of care. Redundant testing is an example of inefficient care as it is costly and time-wasting. The use of an EHRS is associated with keeping test redundancy to a minimum. 

        Studies have found that the use of EHRS resulted in a significant reduction of medical errors. This directly improves patient safety. However, a few studies indicated that the error rate in fact rose, which is discussed in section \ref{ehrs_disadvantages}. % chktex 2

        \paragraph{Organizational outcomes} The billing system provided by an EHRS increases revenue. This is due to better capture and tracking of bills, and a decrease in billing errors. Many costs are also avoided as a result of such a system. To give a two examples of such savings: costs associated with paper and the management of these records, and less redundant testing. Other benefits include better operational performance, improved legal and regulatory compliance, and fewer malpractice claims.

        \paragraph{Societal outcomes} The increased availability of data improves the ability to conduct research. This also helps the public health field in monitoring diseases and the detection of looming outbreaks. EHRS use is also linked to increased physician satisfaction\cite{Menachemi2009}, which in turn improves quality of care.

        \subsubsection{Potential disadvantages}\label{ehrs_disadvantages}

        An EHRS provides several financial benefits. However, it also introduces new financial issues. The adoption and implementation of an EHRS has high upfront cost which involves the purchase and installation of hardware and software, conversion of paper records to digital ones, and user training. Now that EHRS are common, the acquisition costs saw a significant decrease.

        After the acquisition of an EHRS, it needs continuous maintenance. The evolving nature of technology requires hardware replacements and software updates. Consequently, users need to be trained again to adapt to these changes. As mentioned previously, the loss of productivity leads to some revenue loss. The high upfront and the ongoing maintenance costs are considered to be largest barrier to the adoption of EHRS\cite{Menachemi2006}.

        The use of an EHRS may cause some unintended consequences\cite{Campbell2006}, such as an increase in medical errors. Poor system usability, lack of training, and lack of system integration are possible causes of this rise\cite{Koppel2005}. Because of workflow disruptions and adaptation difficulties, an EHRS may evoke negative emotions. To conclude, clinicians may become overdependent on technology. Institutions should still be able to provide care in case there are technical issues.

    % health information exchange
    \subsection{Interoperability}\label{interoperability}

    Health institutions often deploy several systems to satisfy all their requirements \cite{Payne2012}. As a result, data is spread over several repositories which opens up the possibility of data duplication and synchronization issues. If two systems are unable to communicate with each other, then clinicians may resort to paper workarounds as mentioned in section \ref{paper}. Sensory data also needs to find its way into the health record of the patient. If the measuring device has no means to directly send its data to the EHRS, then clinicians are forced to do this manually. Therefore, interoperability is very important in a health care setting. % chktex 2

    %def
    Interoperability is ``the ability of two or more systems or components to exchange information and to use the information that has been exchanged''\cite{IEEE1990}. To go into more detail, interoperability can be divided into four concepts\cite{Benson2016}:
    \begin{itemize}
        \item Technical interoperability: moving data from one system to another.
        \item Semantic interoperability: allow the sender and recipient to understand the same data in the same way without ambiguity.
        \item Process interoperability: when people share a common understanding across a network, systems interoperate, and work processes are coordinated.
        \item Clinical interoperability: the ability for two or more clinicians in different care teams to transfer patients and provide seamless care to the patient.
    \end{itemize}

    \noindent Health care systems and devices should implement standards to achieve interoperability. If every application stores data in its own format of choice, interfaces need to be created for every other applications that wishes to exchange data. This is avoided when systems adhere to a common standard. The next section briefly describes standards, since they are not the main focus of this thesis.

        \subsubsection{Standards}\label{standards}

        When excessive diversity creates inefficiencies and affects effectiveness standards are required\cite{Shortliffe2014}. A hospital contains many independent units spread across primary, secondary, and tertiary care. These units use software best fit for their practice and all record different types of data. For example, the hospital admissions system records patient diagnosis, the pharmacy records prescriptions that were handed out, and the laboratory system records test results. Inevitably, transfer of data between these units is required. 

        To coordinate multiple systems, data must be exchanged. Nowadays too many different systems exist to create point-to-point interfaces for. Standards try to resolve this problem by defining guidelines software system should follow in order to communicate data. An efficient standard requires that data can be easily stored and presented towards the users of an EHRS\@. Also, security measures, such as authentication and access control, need to be interwoven with these standards.

        The use of standards in an EHRS leads to better interoperability which in turn leads to lower development costs. Over time, the continuous addition of proprietary data structures leads to difficult to maintain software and a higher risk of critical bugs. New medical devices and software that comply to standards prevent this. Currently there is no regulator that enforces the use of existing standards, which renders some of them useless.

        One of the most used and implemented standard set is Health Level 7. These message-based standards are created to exchange clinical and administrative data. Another standard such as DICOM is used for the communication and management of medical imaging information\cite{Mildenberger2002}. Finally, standards for medical devices are created by IEEE\cite{Shortliffe2014}.

    \subsection{Usability}\label{usability}



    % definition telehealth
    % opkomst mobiele apparaten
    % interoperability en usability belangrijker dan ooit
    % chronic diseases goed vb voor telehealth/tm
    \subsection{Recent advancements: telehealth}\label{telehealth}

        \subsubsection{Telemonitoring for chronic disease management}



    \subsection{Privacy \& security}\label{privacy}

    Privacy refers to the desire of a person to control the disclosure of personal health and other information\cite{Shortliffe2014}. On the other hand, confidentiality is the ability to control the release of personal health information to a care provider under the agreement that the information will not be spread or used further. Security is the protection of privacy and security, which is achieved through policies, procedures, and safeguards. 

    We can ask several questions related to health information data access and storage: who owns the data? Is it the health provider or the patient? What type and how much of data needs to be stored? Who can read the data? Who can write to the data? Can someone access specific health information without consent? In order to deal with the some of the challenges concerning data access and storage, these questions need to be answered\cite{Meingast2006}.

    An important issue surrounding privacy is medical data access. For example, researching the medical data of a large population can uncover looming threats or an epidemic. Also, disease incidence and intervention analysis benefits from data pooling. However, medical data should not be too accessible. As more and more people gain access to health data of the population, the chance that data is misused or confidentiality is broken increases. Therefore, striking a balance between free information access and protection of privacy and confidentiality is difficult. An argument that promotes free access is to anonymize the data by removing identifiers such as the person's name. However, the individual pieces of information can still reveal the their identity.

    Compared to paper records, electronic records are easily accessible, such as through the internet. This calls for extra security measures to prevent data breaches. If someone wants to access a medical record, this person must first authenticate himself. In turn, the system checks if the person is allowed to access this information. Regardless of the outcome, the audit system logs the access attempt. Such systems need clear rules and policies defined by the institution on who can access what. Another obvious security measure is data encryption.

    To further prevent malicious use of health data, medical staff should follow education and training programs to make them aware of the privacy rules and policies that are in place. In case of privacy violation, the person in question should be punished appropriately.