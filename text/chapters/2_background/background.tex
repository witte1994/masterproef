\section{The Electronic Patient Record}\label{background}

fill

    % nog altijd veel in gebruik
    % signify importance
    \subsection{Moving away from paper}\label{2_ehrs_paper}

    For modern medicine, traditional paper-based medical records are not suited for today's world filled with technology. The drawbacks of information on paper are obvious when compared to digitally stored information.

    \paragraph{Functionality} Digital records allow systems to aggregate, process and create statistics of the data it contains. For example, a system can generate heart rate graphs with statistics, summarizing many values. If the user hovers over a data point in the graph, the exact value is shown. Printed tables and graphs showing the same data lack this interaction. Also, paper records demand more effort from clinicians as data is often spread across multiple dossiers.

    A paper-based medical dossier can store for example medical images, such as x-rays. Compared to a digital image, paper loses a lot of detail. As such, most multimedia types can only be stored digitally and not on paper. Tools to interact with these file types are provided by an EHRS\@.

    In terms of data input, an EHRS can detect and prevent false data input. The system ensures that all necessary data fields are filled in and in the correct format. This results in more complete and accurate data gathering with fewer errors, increasing the information quality.

    \paragraph{Information quality} A summarizing paper noted that the use of EHRS leads to more complete, accurate, comprehensive, and reliable data compared to paper-based records\cite{ehrs_summary}. As mentioned in the previous paragraph, a digital system can impose rules on data fields to avoid missing or entering the wrong data. In terms of comprehensibility, poor handwriting leads to wrong or loss of information, which digital systems avoid. Medical instruments can save measurements immediately into system, avoiding copying by hand.

    \paragraph{Accessibility} Paper records are difficult to access because most of the time only one copy exists. Therefore, transferring these records to other branches or institutions is difficult as the record needs to be located in the often large medical dossier. Also, misfiling, flooding or fires lead to irrecoverable loss of the data. Creating backups of the digital records avoids the last issue, but difficulties surrounding data transfer are still present. Most institutions have their own database structures which hinders the transfer of raw medical data. To remedy this, interfaces are created by the IT staff to interpret and store the data. Sending for example PDF-reports via email is a convenient alternative, although limited in functionality comparable to an actual paper record. Therefore, integration capabilities and the use of standards (section \ref{2_privacy}) are important for an EHRS\@. % chktex 2

    Moving data from paper to a digital format requires a lot of manual work. While automation is possible, such as scanning and processing the paper forms with software, verification is still necessary. All the data fields from the documents need to find a place in the system, which is difficult to achieve. Extra diagnostic information scribbled all over the the document, will be lost during this process. Also, what do we do with unreadable data due to poor handwriting? What happens with two forms which contain partly the same information? Do we save the information twice or do we add extra checks to prevent duplication? Because of these reasons, adopting an EHRS is a significant undertaking for an institution. As such, the benefits are not immediately apparent.

    Because digital storage increases accessibility significantly, security measurements have to be taken. If not, data can be stolen, deleted or even altered, which means a breach of privacy (section \ref{2_privacy}). This also adds to the complexity of developing EHRS\@. % chktex 2

    \paragraph{Efficiency} An important task of an EHRS is to facilitate effective care. An early study saw a 6\% increase in productivity when EHRS were deployed in health care institutions\cite{ehrs_efficiency}. However, other factors are at play, such as the adoption rate. The transition from paper-based documenting to digital is accompanied by a learning curve, which can be steep. After the switch, the productivity will most likely be lower at the start. As the users become more experienced with the software, it will increase. Today most institutions have already made the transition to an EHRS, so this has become less of an issue.

    % component
    \subsection{Functionality}

    \subsection{Extensibility}

    \subsection{Customization}

    \subsection{Privacy}\label{2_privacy}

    \subsection{Standards}