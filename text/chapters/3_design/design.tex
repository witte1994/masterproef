\section{Design}\label{design}

fdsafdsa

    \subsection{Proposal}
    
    The proposal features four core concepts which will be combined to form one cohesive solution: a dashboard, telemonitoring of chronic diseases, customization, and integration. As mentioned in the introduction, multiple applications and integrating them pose several difficulties: data is spread and stored multiple times. Interaction between these applications is initially not present and integration with existing systems requires a significant amount of work from the IT staff of the institution. As more applications are added, this becomes increasingly difficult. To remedy this problem, each application becomes a module which can be plugged into the dashboard system.

    A module-based approach towards the dashboard allows easy customization and integration. Each module will serve its own purpose by showing data for one particular health parameter. As mentioned in section FILL, parameters such as heart rate and blood pressure are often measured to monitor multiple chronic diseases. Integration is handled by the fact that each module is independent and does not require other modules to work properly. This also helps with customization as a clinician can choose which modules are relevant for the patient in question. % chktex 2

    The modules that will be developed are for use in chronic disease monitoring. However, modules can be developed for purposes more fit for in-patient care. Examples are real-time monitoring, documenting diagnoses, and viewing of lab results. Which modules will be implemented and what each one of them tries to achieve is described in section FILl. The implementation section will provide more insight on how this independence is achieved and how future modules can be easily integrated. % chktex 2

    This combination allows the clinician to customize the dashboard according to the needs of each patient, facilitating effective care. Data is found rapidly, albeit summarized or detailed, and not hidden away behind multiple screens. Developers should be able to create modules that will fit into the dashboard, negating the use of multiple applications. The following section describes how the system will work and how the users interact with it.

    \subsection{Dashboard design}\label{2_dashboards}

    Dashboards are used in many contexts. The most common examples include a car speedometer and stock analysis. In computer networking, dashboards are used to quickly interpret network traffic. In our case, we want to gather all relevant health data for a specific patient. To create an effective dashboard, multiple design principles should be kept in mind.
    
    While some dashboards provide fancy graphics, many miss the key point of what they are supposed to accomplish. The primary goal of a dashboard is clear communication, which is achieved through effective design. A formal definition is as follows: ``A dashboard is a visual display of the most important information need to achieve one or more objectives; consolidated and arranged on a single screen so information can be monitored at a glance.''\cite{dashboard}. This definition contains four key characteristics:

    \paragraph{Dashboards are visual displays.} It is important that data is represented visually. Displaying charts and other graphics allows more efficient communication compared to textual information. For example, trends and outliers are easier to spot in a line chart compared to a table with the same values. Instead of using a different visualization for the sake of variety, one should always use the visualization which is best suited to effectively communicate the data\cite{few2005intelligent}.

    \paragraph{Dashboards display information needed to achieve specific objectives.} The data that is shown, must be of use for the job that needs to be done. It is possible that the data needs to be gathered from multiple sources and tailored to the specific context so an efficient visualization can be created.

    \paragraph{A dashboard fits on a single computer screen.} In order to see all information at a glance, scrolling must be prevented. If multiple screens are present, then it is no longer a single dashboard. This leads to another question: what type of display is the information shown on? Due to the wide variety of screen sizes and aspect ratios, a responsive dashboard would be most beneficial.

    \paragraph{Dashboards are used to monitor information at a glance.} Important data should be immediately noticed, whereas very specific details should be hidden. This means data must be summarized or aggregated in order to be effectively shown. However, if the user wishes to view the detailed data, the dashboard should provide means to do so.\bigskip

    \noindent To create an effective dashboard, the user-base must be well known and understood. What type of users are we dealing with? What are their characteristics? These are important questions to ask, as one user may not comprehend one visualization while another one can. This means that the focus should be put on the user\cite{brath2004dashboard}. We can ask the following question to better understand the user's needs:
    
    \begin{itemize}
        \item What metrics does the user need to see?
        \item What context does each metric require to make it meaningful? Do we need to visualize the variance, target to reach, trend\ldots?
        \item What visualization best communicates the metric?
    \end{itemize}

    \noindent On that end, sketches and mockups can significantly help the design process. Multiple iterations, each incorporating feedback from the users, ensure that the end result is satisfactory.

    \subsection{Personas \& scenarios}\label{3_personas_scenarios}

    Based on the proposal mentioned in the previous section, personas and scenarios were created to get a better understanding of its users and how the system will work. Each of these tries to highlight the problems the users face and how the new system tries to solve them.

        % highlights customization and dashboard
        \subsubsection{Jake}

        \paragraph{Persona} Jake is a 25-year-old male who currently works as a nurse at the hospital in his city. He has been working there for four years and lives alone in his apartment. Still being a young adult, Jake grew up with technology. As a result, he experiences no difficulties when using new software on his computer or smartphone. His hobbies include music, playing the guitar, and video gaming.

        Currently, the workflow at the hospital is dated. A new health information system was introduced to summarize and gather all medical data in a singular space. Because this is an all-in-one system, a lot of features that Jake does not need still clutter the screen. Navigating the system is difficult and customization is not present. Jake wishes to only see the features he uses most while hiding the features he does not need.
        
        \paragraph{Scenario} Jake starts his first day using the new system by reading the manual that is accompanied with it. He starts the application and for each patient he is presented with a list of modules that can be added to the dashboard. These modules can help with the following: cardiovascular disease, respiratory disease, diabetes, and others.

        After selecting a few modules, Jake is presented with a dashboard. The dashboard contains all the wanted functionality of the system, where each ``block'' represents a module that was added. Jake moves a few blocks around until he is satisfied with the layout. He noticed a module he does not need anymore and promptly deletes it by selecting the option on the module. After a while, Jake got transferred to another department and wants to add a new module to support his new workflow. He opens the module list and selects the he wants, which is then added to the dashboard.

        By examining the new module Jake quickly notices that on the dashboard a summary is given. But when Jake clicks on the enlarge button, the module expands to fill a larger area where detailed information is shown. Jake feels he is in control of the system and that it will improve his productivity.\bigskip
        
        \noindent This scenario highlights the benefits of customization. Hiding unwanted components, while adding the useful ones allow for a clear dashboard to be displayed. The user is in control and is able to quickly view data of importance. It is still possible to view detailed data.
        
        % highlights customization
        \subsubsection{Dan}

        \paragraph{Persona} Dan is a general practitioner since he graduated from university. He is on the job for 21 years and he is the preferred doctor in his town. Throughout the years Dan has used a multitude of systems and he always tries new ones to improve his workflow. Because he has been a general practitioner for such a long time, he has 500 patients that visit him at least twice a year. Some patients, especially the elderly, visit as much as once a month.

        Most of Dan's patients visit for illnesses such as fever and a cold. To diagnose these illnesses no data is necessarily needed, just a description of the patient's symptoms will suffice most of the time. If Dan performs such a diagnosis, he promptly adds it to the information system and to the electronic health record of the patient.

        However, if a patient visits that has a lot of problems regarding his blood pressure, then Dan has to perform a more complex diagnosis using historical data. In the current system that Dan uses, it is very difficult to search for this data. But what bugs Dan the most is that he has to do it every time this patient visits.

        \paragraph{Scenario} Because of Dan's ongoing curiosity, he tries a new health information system which allows customization for each specific patient. Because the diagnoses of most patients can be very different from time to time, Dan creates a default module group which is displayed for each patient, unless that patient has a specific module configuration. The default module group includes past diagnoses, known allergies, patient information (blood type, last weight, height\ldots), and a current medication list.

        The first patient of the day describes what sounds like a fever. Dan confirms this and hands the patient a prescription. The diagnosis is added to the `past diagnoses' module and the prescribed medication to the `current medication list' module. Dan did not need other health data to perform the diagnosis. Therefore, Dan does not change the configuration of this patient.

        The next patient, an elderly woman, came for her second visit of this month. Dan knows from the past that it will probably be a heart related problem. Dan searches for a `heart' module and adds it to the configuration of the elderly woman. Now both the default module group and a heart rate module are present, which is unnecessary according to Dan. He removes some modules of the default module group. Now, the next time the elderly woman visits, that configuration will be automatically loaded.

        One specific patient had broken his arm three times in less than a year. When the patient came for a routine visit, Dan immediately added a module to easily view x-ray photos and view them in a timeline.\bigskip

        \noindent Again, the benefits of system customization for each patient are obvious. Initially, it will take some time to configure each dashboard for all the patients, but the system will try to make this process quick to complete. Once the configuration is done, the time spent during a consultation will decrease.
        
        % highlights telemonitoring
        \subsubsection{Emily}

        \paragraph{Persona} Just like Dan, Emily has been a successful general practitioner for quite some time. However, she has different needs of an EHRS\@. Between each patient visit, there is a period of time in which Emily does not know what happens to patients regarding certain parameters. For example, if a person has to regularly measure his heart rate and blood pressure because of cardiovascular disease, it is imperative that the doctor is made aware of these values. If Emily sees that these values are not looking well, she can contact the patient to come in for a checkup.

        If a patient has sleep issues, Emily wishes to not see detailed measurement values, but regular descriptions of the night’s sleep. This includes the hours of sleep, amount of wake-ups, subjective feeling of tiredness. Currently, Emily has no way of regularly receiving this information without having the patient visit.
        
        \paragraph{Scenario} Emily recently received notice of a new platform that includes telemonitoring support. Several mobile applications are developed that can send data using an API to the platform, which in turn processes the values and can notify Emily of any anomalies. It includes customization for certain parameters in which Emily can individually assign thresholds for each patient.

        A patient who recently had a cardiac arrest is continuing rehabilitation at home. However, the patient has chest pains and pays Emily a visit. She tells the patient to regularly measure his blood pressure and heart rate, and to take note of these values in a mobile application. This application also allows taking general notes, such as feeling pain or having caught a cold. After they have scheduled the next visit, the patient is sent home.
        
        As the next few weeks pass by, Emily is notified that this patient has crossed a threshold regarding his blood pressure. Immediately Emily checks the measured data and sees a graph of all measured values. This dataset delivers insight into the history of the patient and Emily sees that there is currently no need to panic. She decides to not take action and configures a weekly reminder to keep monitoring the blood pressure.
        
        The next week, Emily receives a notification that the patient has made a note. It reads that he experienced chest pains. Again, Emily takes a look at the blood pressure data and sees a worsening trend leading to hypertension. Emily decides to call the patient to schedule an early visit. The system helped Emily to intervene as soon as the situation seemed to worsen while avoiding having the patient visit too early which in turn saves Emily time.
        
        Another patient has sleep issues. Emily encourages the patient to use a sleep monitor, which is a wristband. This device is connected to a smartphone which communicates the quantitative data to the new platform. The application on the smartphone allows the patient to take note of qualitative data such as a general description of the night or what food he ate. One night the patient slept only three hours and took note that he went out and drank a lot of alcohol. This could explain for example the bad sleeping rhythm for the next few days. Emily wishes to keep track of this patient on a weekly basis and configures the platform to notify her.\bigskip

        \noindent Here, the effects of telemonitoring are highlighted. By generating reports and alerts, the caregiver can intervene when necessary. Combine this with the aforementioned dashboard platform, greater efficiency of care can be accomplished.

        % highlights integration
        \subsubsection{Anna}

        \paragraph{Persona} Anna is a software engineer working at the local hospital. From the early 2000's, she was responsible for the integration of health software systems. As such, she has experienced the continuous change associated with health software. Each unit of the hospital uses different software, while they all share a general system to view patient records. The interaction between these applications is one of Anna's responsibilities.

        As time goes on, new software and medical instruments that improve workflow are purchased. Sometimes these replace old systems, otherwise they are added to work alongside them. Both are becoming increasingly difficult to do, as the codebase of the EHRS grows. When an older system gets replaced, Anna has to weed through the code and delete little pieces, hoping other parts wont break. It is a frustrating task as often band-aid solutions are applied.

        \paragraph{Scenario} A new EHRS was installed that should improve integration of other applications. Anna skims through the documentation and starts the transition process. She quickly notices that for each device or application she needs to create a back end data structure and a module that represents it in the EHRS\@. The new EHRS is designed that these modules can serve as complete standalone applications or be part of a larger set of modules.

        After the initial transition period, everything is put into place. Not soon after, new bedside monitors with accompanying software were purchased. These replace the old bedside monitors and consequently, the software. Thanks to the new EHRS, Anna deletes the old module without much hassle. Also, she reworked the back end data structure to work with the readings from the new device. The data of the old device is ported to the new data structure, so no data is lost. Anna designed the new module to be similar in appearance and functionality. After implementing the module, the staff barely notices any change and workflow resumes with minor disruption.\bigskip

        \noindent This persona and scenario highlight the importance of integration. Significant time and effort can be saved if this process is simplified. Also, maintenance is easier due to looser coupling of the modules.

    