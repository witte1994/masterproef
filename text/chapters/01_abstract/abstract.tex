\section*{Abstract}
\addcontentsline{toc}{section}{Abstract}

Electronic health records have transformed the way care is delivered. Compared to the old paper medical record, there are many benefits to be gained. These benefits include a higher quality of care, increased adherence to medical guidelines, and a reduced amount of medical errors. Also, costs can be saved. The rise of mobile technology can further benefit the health care sector by allowing care delivery from a distance.

There are also disadvantages associated with digital health systems. One example are its high acquisition and maintenance costs. Other cost increases are associated with the training of the medical staff to work with the software. This adaptation period is responsible for a loss in productivity. Another unintended consequence of these systems is an increase in medical errors. A prototype was designed to tackle these last two drawbacks in particular.

A lack of system interoperability and good usability are to blame for some of the drawbacks of electronic health record systems. The ability of a system to communicate with other systems is defined by its interoperability, whereas usability refers to the ease of use and learnability of a system. The design of the prototype describes a modular dashboard wherein the clinician can freely choose its functionality according to his/her workflow. Interoperability was provided by having a loosely coupled back end and a base component to speed up front end development. Several principles defined by literature were kept in mind to provide good usability.

Six individuals participated in a usability test of the prototype. The participants were spread across several health care units, such as nursing and emergency response. The results indicated that the dashboard had good usability, together with many opportunities to improve. However, the interoperability aspect could not be tested. Based on these results, several topics of future work were highlighted.