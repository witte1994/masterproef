\section{Introduction}

As of today, most health institutions use a digital system to manage health records. These systems are designed to improve the quality of care on many levels compared to paper-based records. To start, digital systems help with the capture of accurate and complete data. Devices communicate directly with the system to store its data. Also

Many of the issues surrounding paper-based medical records are solved after the adoption of a digital system. As a person ages, its medical dossier grows. This 

In this thesis, we explore the effects of electronically stored medical data on the health care industry. Along with benefits come drawbacks and we propose a solution to some of the unintended consequences associated with electronic health record systems.

% ehr rise,
% solves issues surrounding paper
% other considerations
% quickly moving due to tech: TM
% raises some issues of its own

% in this thesis we highlight some of these issues and propose a solution

% go over thesis structure