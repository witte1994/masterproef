\section{Introduction}

As of today, most health institutions use a digital system to manage health records. These systems are designed to improve the quality of care on many levels compared to paper-based records. The recent advancements of mobile technology can improve the care quality even further.

% benefits
To start, digital systems help with the capture of accurate and complete data, increasing its quality. Devices communicate directly with the system to store new data values. Also, manual data input can be checked against several rules to ensure it is correct and complete. This kind of error prevention is not possible on paper.

The storage of medical information in a digital format is also much more convenient. First, it is easier to create a back up of the record, preventing data loss. Second, the data is easier accessed. It is essential that the required data to provide care is always available. This can not be guaranteed for data stored on paper, as it is only available on one location at a time. Increased accessibility also benefits the public health sector. For example, a large data set can be built from several repositories, which may uncover a looming epidemic.

In term of security, a digital system can automatically log all access attempts and changes made to a record. For paper records, this is significantly more difficult to log. Also, an access control system can enforce strict privacy rules that determine who gets access to what data. As soon as a paper dossier leaves its secure storage facility, the privacy is in the hands of the person who retrieved it.

Technological advancements have increased the feasibility of providing care at a distance. For example, a smartwatch can record the heart rate of a person and send it to the caregiver on a regular basis. If there are any abnormalities, then the caregiver can contact the patient to schedule a visit. This prevents the scheduling of regular visits to perform these checkups in person, saving resources of both the patient and the caregiver.

% drawbacks, considerations
Based on these benefits, it is without question that the electronic health record is a necessity to provide high quality of care. However, there are also several drawbacks related to digitalization. The transition from paper to digital records is a very costly endeavour. These costs include staff training, the process of converting the paper records to a digital format, and the acquisition and maintenance of the system. Also, the chance that a data breach occurs increases as the records are easier accessed. There are also other concerns such as interoperability. How do we communicate data from one device or system to the other? Is data conversion needed or is a common standard in use?

In this thesis, we explore the effects of electronically stored medical data on the health care industry. Based on our findings, we propose a solution to some of the negative consequences associated with electronic health record systems. This text describes the process of the design, implementation, and evaluation of this solution.

% background
First, we present a thorough background of the electronic health record. It covers topics such as its components and the impact it has on the health industry. Hereafter, we discuss software interoperability and usability, which are very important topics in the context of health care. The recent advancements of mobile technology made remote care delivery a more feasible alternative to traditional care delivery. This evolution is of importance for future health systems, which we discuss in the telemonitoring section. To conclude this chapter, the last section describes the implications of digitally stored records on data privacy and security.

% design
The next chapter explains the design process. First, we specify a solution based on the issues mentioned in the background chapter in more detail. Hereafter, a development framework and dashboard design principles are explained. This led to the creation of personas and scenarios. These artifacts illustrate how our solution will work in practice. Before the development of the prototype started, a concrete specification of the application was created which describes its structure and components.

% implementation
The start of the implementation phase involved choosing the frameworks and libraries best suited for our application. First, we describe the reasoning behind these choices, whereafter the structure and components of both the back and front end are explained. To showcase the result, we provide screenshots for all the components of the front end.

% evaluation and conclusion
After the development of the prototype ended, a usability test was designed. We explain the purpose of such a test and who it targets. Hereafter, the concrete structure of the test is specified. The next chapter describes the recruitment process and the results of the usability test. Based on these results we draw conclusions which indicate several topics of interest for future work.
