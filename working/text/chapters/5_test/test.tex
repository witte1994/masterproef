\section{Usability test}\label{usabilitytest}

As the prototype neared completion, a usability test was designed. As the name implies, the purpose of such a test is to assess the usability of a prototype. While the prototype is being tested by a user, the researcher takes note of any usability issues that may arise. This evaluation method results in valuable feedback, as it shows directly how real users use a system~\cite{Nielsen1993, Lazar2017}.

In this chapter we describe the design and results of a usability test to assess the user experience of the dashboard prototype. This test will focus on the usability aspect of the prototype, and not on the extensibility. However, in the future work section (\ref{future_work}) we give suggestions to test this aspect. All the documents created for the test are found in appendix~\ref{appendix_test_docs}.

    \subsection{Purpose}

    The goal of this test is to evaluate the usability of the dashboard prototype. The modules were designed with the usability principles mentioned in section~\ref{usability} in mind. Customization was added by giving the user more freedom in defining the functionality present in the dashboard. However, the experience of the end-users will determine if this is of added value when used in a clinical setting. Therefore, we are interested in gathering the following information:
    \begin{itemize}
        \item Is the dashboard easy to use? Is there a low or high risk of errors?
        \item Is the dashboard easy to navigate? Are all functions easily found?
        \item Does the dashboard as a whole succeed in displaying information in a clear and concise manner? How about the individual modules?
        \item Is the layout customization practical? Is it fast, slow, error-prone\ldots?
        \item How does the usability of the prototype compare to the system the tester uses?
    \end{itemize}

    \noindent Depending on the answers we receive to these questions, we can determine the next steps to take in future research, mentioned in section~\ref{future_work}. While our primary concern is the evaluation of our prototype, we also want to gain insight on the EHR system(s) the tester uses. This way, we can identify potential improvements or additional features for our prototype going forward. % chktex 36

    \subsection{Participants}\label{test_participants}

    The prototype is designed for use in different medical settings, which are staffed by many different roles. To get the best possible results, we want our group of participants to cover as many different roles as possible. For example, an general care nurse has completely different expectations from an EHR system compared to a paramedic. This will yield feedback we would otherwise not have received if all participants had the same role.

    Therefore, we need variety. Our participants should represent a wide spectrum of the health care industry in terms of the roles they fulfill. On top of that, having participants from different age groups is encouraged. Given the scope of this thesis, we chose the following sectors and roles to gather participants from:
    \begin{itemize}
        \item Laboratory unit: conducting tests and management of lab orders and results.
        \item Intensive care unit: requires close/real-time monitoring of several patients.
        \item General inpatient care: for example non-critical pediatric care within a hospital.
        \item Emergency response: for example a paramedic. Needs to respond fast, handle stressful situations, and needs to be mobile.
        \item A general practitioner: sees many patients daily with a large variety of conditions.
    \end{itemize}

    \noindent Now that we know who we want to test, the next question is how many. Nielsen determined that five participants find almost as much usability problems as for example a group of ten~\cite{Nielsen2012,Lazar2017}. In other words, five participants result in the optimal benefit-cost ratio associated with user testing. However, there are exceptions to this rule. A quantitative study for example should test a lot more individuals in order to get statistically significant results.

    Our test is qualitative in nature, since we want to get a deeper understanding of what the needs of the end-user are. We are interested in \emph{why} the user experiences certain things. Therefore, according to Nielsen, we don't need to gather many participants, but preferably still more than five. This is due to the fact that the end-users of the dashboard have different professions and work in different care settings. The needs of each participant vary greatly between these settings, which is why five participants may fall short in uncovering some important usability issues.

    \subsection{Test procedure}

    During the recruitment, willing participants may choose when and where the test will take place, given the location will not cause any disturbances. After meeting up, the following procedure takes place, with the estimated time for each step:
    \begin{myenumerate}
        \item Pre-test, 5 to 10 minutes:
        \begin{myenumerate}
            \item Greet and brief the participant.
            \item Ask for informed consent.
            \item The participant fills in a questionnaire which asks for demographic information and information concerning the EHR system he/she currently uses.
        \end{myenumerate}
        \item Test of the dashboard prototype, 20 to 25 minutes:
        \begin{myenumerate}
            \item Give the participant a brief tour of the prototype.
            \item The participant starts following the step-by-plan while being observed. No help or suggestions will be given unless necessary.
        \end{myenumerate}
        \item Post-test, 10 minutes:
        \begin{myenumerate}
            \item The participant fills in a second brief questionnaire about the user experience.
            \item The participant is interviewed. We go over suggestions, what was good/bad, future modules\ldots Additional questions may be asked as the interview goes on.
            \item Debrief the participant and hand out reward.
        \end{myenumerate}
    \end{myenumerate}

    \noindent The informed consent form found in appendix~\ref{appendix_test_consent}, is printed twice and filled in on the spot. The two questionnaires discussed in the next section, were created in Google Forms and are filled in using the laptop which has the prototype installed. Finally, the prototype is tested via the Google Chrome browser. A backup of the database is restored between each test to ensure that every participant will work with the exact same data.

    \subsection{Data gathering}

    The test results in quantitative data from two questionnaires. However, the qualitative information is more important, which is gathered from observing the participant during the test of the prototype and from the interview afterwards. Both questionnaires are found in appendix sections~\ref{appendix_pretest} and~\ref{appendix_posttest}.

    \paragraph{Quantitative data} The first questionnaire asks at the start for demographic information of which the following can be qualified a quantitative data: 
    \vspace{-6pt}
    \begin{myitemize}
        \item Age.
        \item Technology experience (Likert scale, 1: not experienced, 5: experienced).
    \end{myitemize}
    \noindent The second part asks for information regarding the current EHR system that the participant uses: 
    \vspace{-6pt}
    \begin{myitemize}
        \item Satisfaction (Likert scale, 1: very dissatisfied, 5: very satisfied).
        \item User friendliness (Likert scale, 1: strongly disagree, 5: strongly agree).
        \item Supports the participant's needs (Likert scale, 1: strongly disagree, 5: strongly agree).
    \end{myitemize}

    \noindent The second questionnaire asks for ratings regarding the usability of the prototype: 
    \vspace{-6pt}
    \begin{myitemize}
        \item General experience (Likert scale, 1: very bad, 7: very good). Because this is the general opinion of the participant, more steps are defined for this scale.
        \item Easy to use (Likert scale, 1: strongly disagree, 5: strongly agree).
        \item Dashboard looks clean (Likert scale, 1: strongly disagree, 5: strongly agree).
        \item Modules look clean (Likert scale, 1: strongly disagree, 5: strongly agree).
        \item Modules served as a good example (Likert scale, 1: strongly disagree, 5: strongly agree).
        \item Enough customization options present (Likert scale, 1: strongly disagree, 5: strongly agree).
    \end{myitemize}

    \noindent No other quantitative data is gathered. During the observation, errors are noted, but not counted.

    \paragraph{Qualitative data} The bulk of the qualitative data is gathered during the test of the prototype by observation and afterwards during the interview. When the participant is following the step-by-step plan, the following notes may be taken:
    \vspace{-6pt}
    \begin{myitemize}
        \item The participant struggled finding UI element X.
        \item An error was made regarding X.
        \item The participant hesitated at step X.
        \item The participant asked question X regarding Y\ldots
    \end{myitemize}

    \noindent Questions related to the usability of the prototype and the current EHR system, current paper use in their care setting, module suggestions\ldots are asked during the interview. Section~\ref{results_interview} describes what qualitative data was retrieved.

    \subsection{Test scenario}

    During the step-by-step process, the participant works with the dashboard of three fictional patients. A short background is given for each patient and participant performs actions which relate to the situation of the patient. The participant has seen all aspects of the dashboard after the test of the prototype is complete. Appendix~\ref{appendix_test_steps} contains the step-by-step process. Also, the screenshots in appendix~\ref{appendix_test_screens} show the dashboard at the start of each scenario. We now describe each scenario.

    \paragraph{Patient 1} Background: Kenny is an account manager of a large accounting firm. The combination of his stressful job, sedentary lifestyle, and smoking habit have lead to health issues. He has frequent palpitations and high blood pressure. At home, Kenny has to measure his weight, heart rate, and blood pressure on a regular basis with devices that send the values to his electronic patient record. In this scenario the participant does the following:
    \vspace{-6pt}
    \begin{myenumerate}
        \item Apply a filter on the history module.
        \item Add and configure a telemonitoring module to fill the empty space of the dashboard.
        \item Reorder the modules on the summarizing panel, while adding and configuring a small telemonitoring module.
    \end{myenumerate}

    \paragraph{Patient 2} Background: Since Jozefien retired, she has been gaining weight at an alarming rate. As a countermeasure, she regularly visits her general practitioner. Each consultation, the practitioner updates her medication scheme and tells her what she needs to pay attention to. In this scenario the participant does the following:
    \vspace{-6pt}
    \begin{myenumerate}
        \item Remove and add a prescription, check for interactions.
        \item Remove the allergy module to replace it with a workflow module.
        \item Edit some steps of a workflow.
        \item Reset the checklist and add some new tasks to it.
        \item Hide a parameter in the small telemonitoring module found in the left panel.
    \end{myenumerate}

    \paragraph{Patient 3} Background: After a serious car accident, Bert is hospitalized in the intensive care unit. Bert has many allergies, an elaborate medication scheme, and misses some vaccinations. Currently he is being treated by several clinicians and being observed by nurses. It is important that everyone is aware of these allergies, the medication scheme, and missing vaccinations. However, the data in the EHR system is not up to date. The latest most up to date medical data is still stored in a paper dossier. In this scenario the participant does the following:
    \vspace{-16pt}
    \begin{myenumerate}
        \item Change and add an allergy.
        \item Remove and add a vaccination.
        \item Apply a filter on the history module.
    \end{myenumerate}

    \subsection{Recruiting participants}

    The search for participants began immediately after the usability test design was completed. A total of 9 people were contacted with the following backgrounds:
    \vspace{-14pt}
    \begin{myitemize}
        \item 2 nursing students, both had internship experience in a hospital setting.
        \item 1 elderly care nurse.
        \item 2 intensive care unit nurses.
        \item 1 clinical laboratory department head.
        \item 1 paramedic.
        \item 2 home nurses, working for the same practice.
    \end{myitemize}

    \noindent Five individuals immediately were scheduled to test the prototype. The two home nurses and one nursing student declined due to time constraints. The paramedic expressed interest and was scheduled several weeks later. Four general practitioners were contacted, but were unable to participate due to the short notice of the study. Six participants have tested the prototype.

    \subsection{Results}

    Before every test, the database of the application was reset and the log in functionality was tested to avoid any technical problems. As a result, every test ran smoothly without any issues. Every participant tested the prototype in a secluded and silent environment. Four tests were conducted at the homes of the participants and the other two in empty classrooms of Hasselt University. The test was estimated to take approximately 40 minutes to complete. In anticipation of lengthy interviews, a hard stop was put in place should the test pass the 60 minute mark. This was the case for two tests. The durations of the other four were between 45 and 55 minutes. This caused no issues for any of the participants. When the test concluded, participants were rewarded with a bottle of wine. During the recruitment phase, each contacted individual was aware of a reward, but they did not know what it was beforehand. % chktex 1

    \subsubsection{Pre-test questionnaire}

    Table~\ref{table:pre-test} shows some general information from the pre-test questionnaire regarding the participants. It should be noted that 5 out of 6 participants are between the ages 21 and 25. They also indicate a higher familiarity with technology, compared to the older participant. The younger participants also have limited work experience. The pre-test questionnaire also indicated that all participants use their smartphone and lap-/desktop daily. Four participants use their tablet a few times every month, of which one participant uses it daily. One participant uses a smartwatch daily, and is the only one to use a smartwatch at all.
    
    \begin{table}[!t]
        \resizebox{\textwidth}{!}{%
        \begin{tabular}{ccllc}
            \hline % chktex 44
            \textbf{Age} & \textbf{Gender} & \textbf{Profession}        & \textbf{Experience} & \textbf{\begin{tabular}[c]{@{}c@{}}Tech\\ experience (1-5)\end{tabular}} \\ \hline % chktex 44
            21           & M               & Nursing student            & 4 years internship  & 4                                                                  \\
            25           & F               & Elderly care nurse         & 1,5 years           & 4                                                                  \\
            23           & F               & Intensive care unit nurse  & 1 year 5 months     & 4                                                                  \\
            23           & F               & Intensive care unit nurse  & 5 months            & 4                                                                  \\
            49           & F               & Laboratory department head & 28 years            & 3                                                                  \\
            22           & M               & Paramedic                  & 1 year              & 5
        \end{tabular}%
        }
        \caption{Pre-test results: general participant information}\label{table:pre-test}
    \end{table}

    Table~\ref{table:pre-test-ehr} shows that all participants use a different EHR systems for their care setting and only one of them is used on a mobile device. The user satisfaction of the EHR systems scored an average of 4 out 5, with user friendliness scoring a bit less. The participants were more neutral towards the features the EHR system provides, scoring a 3,17 on average. Only two EHR systems provide customization. Lastly, one participant uses no less than 6 applications in combination with the EHR system, while two participants use none. During the interview, more questions were asked concerning their current EHR system.

    \begin{table}[!t]
        \resizebox{\textwidth}{!}{%
        \begin{tabular}{lcccccl}
        \hline
        \textbf{EHR system} & \textbf{Mobile} & \textbf{\begin{tabular}[c]{@{}c@{}}Satisfaction\\ (1-5)\end{tabular}} & \textbf{\begin{tabular}[c]{@{}c@{}}User friendliness\\ (1-5)\end{tabular}} & \textbf{\begin{tabular}[c]{@{}c@{}}Complete\\ toolset (1-5)\end{tabular}} & \textbf{\begin{tabular}[c]{@{}c@{}}Custom-\\ izable\end{tabular}} & \multicolumn{1}{c}{\textbf{\begin{tabular}[c]{@{}c@{}}\# other \\ systems\end{tabular}}} \\ \hline
        Orbis               & No              & 3                                                                     & 3                                                                          & 2                                                                         & No                                                                & 0                                                                                        \\
        GEMS                & No              & 3                                                                     & 3                                                                          & 3                                                                         & No                                                                & 6                                                                                        \\
        Metavision \& GEMS  & No              & 5                                                                     & 4                                                                          & 4                                                                         & No                                                                & 2+                                                                                       \\
        ICCA                & No              & 5                                                                     & 5                                                                          & 4                                                                         & Yes                                                               & 4                                                                                        \\
        HIX                 & No              & 4                                                                     & 4                                                                          & 3                                                                         & Yes                                                               & 1                                                                                        \\
        KWS                 & Yes             & 4                                                                     & 3                                                                          & 3                                                                         & No                                                                & 0
        \end{tabular}%
        }
        \caption{Pre-test results: EHR systems in use by participants. Note: same order is respected as table~\ref{table:pre-test}.}\label{table:pre-test-ehr}
    \end{table}

    \subsubsection{Prototype test: observations}

    All participants were able to complete the test within the allotted time of 20 to 25 minutes. The two intensive care nurses were noticeably faster in completing all the steps, nearing the 15 minute mark. The elderly care nurse and the laboratory head were often unsure on where to click and often asked questions instead of trying. This may explain the fact that they took a bit longer to complete the test compared to the other participants. However, every participant received the same brief tour of the prototype, so no participant had more knowledge of the prototype compared to the others. We now go over the comments made by the participants during the test.

    There were several comments regarding the prescription module. One participant noted that it would be useful if empty dosage fields were automatically filled with zeroes. Currently, the user needs to fill all fields manually. Another participant suggested something similar. Because today's date had to be filled in multiple times throughout the test, the participant would've liked a button that would automatically fill it in. This participant stressed that error checking was a very important component of the systems they currently use. Regarding the end date of prescriptions, one participant noted that they sometimes have to administer medication indefinitely, which the module does not support at the moment. In case there are interactions between medicines, one participant noted that it would be helpful to see the actual interaction effects in addition to the interacting medicine.

    The test highlighted two issues. First, four participants struggled to add a new task to the checklist. This was done by opening the context menu on top of the checklist description. In this case, a button would be a better option. Second, three participants had difficulties finding the option to add a vaccination entry, which again was found in a context menu. More careful thought must be given on when to use these context menus. However, one participant had no issues with the context menu and thought it was really useful. This participant shared this thought also for the date picker.

    Participants had a few more comments. Instead of clicking the arrow button to open the dashboard of a patient, the user can click anywhere in the row of the table to open it. Also, one participant liked both the normal and small telemonitoring modules in particular, describing them as very clean and simple. To conclude, a participant described the importance of access control and privacy surrounding the patient data history module. 
    
    \subsubsection{Post-test questionnaire}

    Table~\ref{table:post-test} shows the results of the Likert scale questions concerning the usability of the prototype. The results are very positive, indicating a good overall experience with the dashboard. Furthermore, all participants indicated that they had no trouble finding their way around the dashboard. However, these results are not conclusive, due to the small sample size. Also, the fifth participant was the only older person in the sample group. The general experience of the dashboard and the experience with technology were both lower for this participant. But then again, a laboratory setting does not share many similarities with the other settings found in the sample group. 
    
    Maybe the most important result is that the modules served as good examples of what is possible with the dashboard. This allowed the participants to better relate the prototype to their work setting, which may lead to valuable feedback during the interview. Also, the questionnaire asked the participants to suggest some modules which would fit the dashboard, which we discuss in more detail in the next section.

    \begin{table}[!t]
        \resizebox{\textwidth}{!}{%
        \begin{tabular}{cccccc}
        \hline
        \textbf{\begin{tabular}[c]{@{}c@{}}General\\ experience\\ (1-7)\end{tabular}} & \textbf{\begin{tabular}[c]{@{}c@{}}Ease\\ of use\\ (1-5)\end{tabular}} & \textbf{\begin{tabular}[c]{@{}c@{}}Dashboard\\ clear UI\\ (1-5)\end{tabular}} & \textbf{\begin{tabular}[c]{@{}c@{}}Modules\\ clear UI\\ (1-5)\end{tabular}} & \multicolumn{1}{l}{\textbf{\begin{tabular}[c]{@{}l@{}}Modules good\\ examples (1-5)\end{tabular}}} & \textbf{\begin{tabular}[c]{@{}c@{}}Enough\\ Customization\\ (1-5)\end{tabular}} \\ \hline
        6                                                                             & 5                                                                      & 5                                                                             & 5                                                                           & 4                                                                                                  & 5                                                                               \\
        6                                                                             & 4                                                                      & 3                                                                             & 4                                                                           & 4                                                                                                  & 4                                                                               \\
        6                                                                             & 4                                                                      & 5                                                                             & 4                                                                           & 5                                                                                                  & 5                                                                               \\
        7                                                                             & 5                                                                      & 5                                                                             & 5                                                                           & 5                                                                                                  & 5                                                                               \\
        4                                                                             & 3                                                                      & 4                                                                             & 4                                                                           & 4                                                                                                  & 4                                                                               \\
        7                                                                             & 5                                                                      & 5                                                                             & 5                                                                           & 5                                                                                                  & 5                                                                               \\ \hline
        \textbf{6}                                                                    & \textbf{4,33}                                                          & \textbf{4,5}                                                                  & \textbf{4,5}                                                                & \textbf{4,5}                                                                                       & \textbf{4,67}                                                                   \\ \hline
        \end{tabular}%
        }
        \caption{Post-test results: questionnaire to assess usability of the dashboard. The bottom row indicates the average scores. Note: same order is respected as table~\ref{table:pre-test}.}\label{table:post-test}
    \end{table}

    \subsubsection{Interview}\label{results_interview}

    Several topics were brought up during the interview, such as potential modules to add to the dashboard, use cases for the current modules, the current EHR system that is in use, the existence of paper in the work setting, what was good or bad, and other suggestions. For each topic, we discuss the feedback the participants gave us.

    \paragraph{Module suggestions} Two participants suggested a drip management module. A drip, or intravenous therapy, is the administration of a fluid solution directly into a vein. This is often done via syringes. The module would help regulate and track the volume of the drip. Also, if the drip administers a medicine which affects blood pressure, then the module tracks this parameter as well. In case something is unusual is happening to this parameter, the module alerts the caregivers. This module is much more practical than managing the drips of multiple patients the manual way. A computer can easily track many ongoing drips.

    Another module was suggested twice. Fluid balance is the balance between fluid that enters the body and fluid that leaves the body. This closely relates to drips, which serves as fluid input. A fluid balance module helps tracking the amount of fluid that enters and leaves a patient's body. In case any abnormalities occur, then the module can notify the clinician. This again is an example how these systems can automate several steps of these processes.

    Wound management is tedious process, as no wound is the same. A cut needs to be treated different than a stab from a needle. There exists a large variety of these wounds that each have their own step-by-step guide to treat them. A wound management module can provide the right guide at the right time. This way clinicians don't need to search through a thick file. Also, a participant suggested that if the module was used on a smartphone, the camera can be used to take pictures of the wound.

    A participant also suggested an emergency contact module. For example, this module contains information of the family members of the patient. But it may also contain the contact information of the doctor on night duty. Since these doctors rotate their night shift periodically, it takes time look up the information manually via a paper calendar. If this again was used on a mobile phone, calls can be directly initiated from the module.

    Other suggestions included body temperature monitoring, diabetes management, diet management, and pain status or relief modules. However, the previously mentioned ones were thoroughly explained by the participants on how they would see it being used in practice. One participant suggested a notification hub. This would be very beneficial for the dashboard, but also difficult to realize. A notification hub is not a module, but it is deeply rooted in the architecture of the application. From our approach, all integrated modules need to have a standardized way to implement the firing of notifications.

    \paragraph{Use cases current modules}

    Every participant was asked if they could devise a scenario in which they would use the dashboard or a module from it. This resulted in several use cases, in particular to the workflow and checklist modules. Where one participant mentioned wound management as a potential module, another participant would use the workflow module to do this. In this case, every specific wound represents a workflow. 
    
    Another participant mentioned that every operation leads to a specific care path that needs to be followed. The participant saw this as an opportunity to model these paths in the workflow module. Furthermore, another workflow use case was to store patient specific requests in there. The participant gave following example: ``An elderly lady wants to visit the chapel of the hospital every Sunday morning. However, she may not go alone and someone needs to accompany her.'' The workflow module serves as a reminder tool in this case.

    If the workflow module allowed branching in the form of ``if A happens, do B, else C'', then a lot more possible use cases would arise, according to a participant. Another suggestion was to extend the patient information module by logging the ``Do Not Resuscitate'' order, or in other words, allow natural death to occur. To conclude, the checklist module also generated some use cases, such as: record the current administered medications, log the steps of the morning care routine, checking of parameters, and wound management.

    \paragraph{Current EHR system}

    Four out of six participants mentioned that their current system is not customizable. When the other two participants were asked to clarify the customization aspect in their EHR system, they both mentioned that ``a doctor gets a very different layout compared to mine'' and that the IT staff listens to their feedback. This is not what we meant by customization, which turned out to be limited in those two systems as well.

    Some participants used a few software packages next to their EHR system. For one participant, the functions of these extra software packages were as follows: a consultation and transportation planner for patients, a diet planner which was connected to the catering service, and a support service for technical issues. One participant noted that next to their EHR system, they used a software package to retrieve lab results and send lab orders. When taking blood samples, these had to be recorded in both applications, which seemed useless to the participant.

    Several other issues were highlighted regarding the current EHR systems. In one case, the system changes often, which causes it to break occasionally. Also, whenever an external digital document was received, the records it contained needed to be entered manually into the EHR system. Another participant highlighted that the system required a significant amount of training and that the application was not easy to use.

    Lastly, a participant emphasized the fine-grained privacy rules and access control measures that were in place in their system. However, this resulted in some cases that there was no access to a medical record when there should be, hindering care delivery. The participant concluded that striking a balance between data availability and strict privacy rules is a difficult subject for such software. Another participant noted that the EHR system gave away too much information, thinking that they should not be able to see some pieces of information.

    \paragraph{Paper usage}

    Generally, paper usage was fairly limited for all participants. The intensive care nurses used no paper for internal care delivery. However, lab, prescription, and imaging orders were transferred on paper to other departments. The laboratory manager does not use any paper anymore since the latest upgrade of their EHR system. One participant mentioned that the administered medication for patients was still written down on paper, which seems troubling.

    To conclude, one participant made the interesting comparison of paper usage with the use of laptop carts. These carts make it possible for clinicians to have more mobility in terms of computer access. However, the participant noted that these carts would get lost or get in the way, and wanted a truly mobile solution such as tablet computers. This participant noted as well that tablets may solve some other inefficiencies. When recording health parameters of patients, then the clinicians might be required out of protocol to immediately enter the data into the digital system. This results in a lot of walking from room to room which can be avoided by having a mobile device available to directly record the data with.
    
    \paragraph{General}

    One participant described that the customization aspect will be helpful to remind the clinician with what patient they are dealing with. Currently, the same layout is displayed for each patient, which may lead to confusion during a high workload. However, the same participant and another one, felt that the customization would be a lot of work for patients that wont stay long. A solution for this, suggested by three participants in total, was to provide default module configurations, either provided by the system or created by the clinician. One participant mentioned that it should be possible to prohibit the removal of some modules that are absolutely necessary in a given setting. The laboratory manager did not see the added benefit of customization in their work setting. Mainly, because their workflow does not frequently change, and because the software they currently use is very satisfactory.