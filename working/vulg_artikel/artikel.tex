\documentclass{article}
\usepackage[dutch]{babel}


\title{Vulgariserend artikel\\---\\The Modular Electronic Health Record Dashboard: A Novel Approach to Interoperability and Usability}
\date{Januari 2019}
\author{Dennis Cardinaels}

\selectlanguage{dutch}
\begin{document}


\maketitle

\section{Het elektronisch pati\"{e}ntendossier} 

Iedereen die in de laatste jaren de huisarts een bezoekje heeft gebracht is er mee in aanraking gekomen: het elektronisch pati\"{e}ntendossier. Dit digitaal dossier laat geneeskundigen, dus niet alleen huisartsen, toe om effici\"{e}nter zorg te kunnen verlenen. Vroeger werden de meeste medische gegevens opgeslagen in een papieren dossier. Dit dossier bevat bijvoorbeeld laboresulaten, r\"{o}ntgenfoto's en observaties van de huisarts.

Er zijn echter veel nadelen verbonden aan het opslaan van deze gegevens op papier. Zo kan deze data moeilijk doorgegeven worden aan andere geneeskundigen. Ook kunnen deze gegevens permanent verloren gaan door bijvoorbeeld verduring en natuurlijke fenomenen zoals een overstroming. Daarbij kunnen gegevens onvolledig of onleesbaar zijn vanwege een slecht geschrift. Dit heeft beduidende gevolgen voor het effectief verlenen van zorg, zoals een verhoogde kans op medische fouten. Hier moet het elektronisch pati\"{e}ntendossier een oplossing voor bieden.

Het digitaal opslaan van medische gegevens heeft tal van voordelen. Ten eerste laat het medische instituties toe om kosten te besparen. Een voorbeeld hiervan zijn de kosten betrokken met het printen van papier. Nog een voordeel is dat medische gegevens correcter en makkelijker beschikbaar zijn. Dit zorgt voor betere zorgverlening. Ook voorkomt een digitaal systeem dat een geneeskundige bepaalde fouten maakt.

Het elektronisch pati\"{e}ntendossier heeft echter ook enkele nadelen. De prijs van het aanschaffen en onderhouden van een digitaal systeem is zeer hoog en wordt gezien als het grootste knelpunt voor medische instituties. Deze kost is echter aan het dalen naargelang deze systemen meer in gebruik worden genomen door de medische industrie. Een onverwacht neveneffect is dat er meer medische fouten worden gemaakt vanwege het gebruik van een digitaal systeem. Hier is een slechte gebruikservaring voor verantwoordelijk. Ook daalt de productiviteit na de installatie van het systeem omdat de geneeskundigen er aan moeten wennen.

Dit toont aan dat er nog ruimte is voor verbetering. Als reactie hierop is er een prototype ontwikkeld met als doel de reeds gemelde problemen aan te pakken.

\section{Het medisch dashboard}

De volgende twee onderwerpen zijn verantwoordelijk voor de voorafgemelde problemen: \emph{bruikbaarheid} en \emph{interoperabiliteit}. Als het systeem moeilijk bruikbaar is, of in andere woorden niet gebruiksvriendelijk is, dan verhoogt de kans dat er fouten worden gemaakt. Ook leidt dit tot frustraties omdat de gebruiker langer moet wennen aan het systeem. Stel dat een geneeskundige meermaals tussen twee programma's moet wisselen met elk een compleet andere interface, dan wordt het alleen maar erger.

Dit leidt naar het tweede onderwerp: interoperabiliteit. Het is belangrijk dat meerdere systemen en medische toestellen goed kunnen communiceren met elkaar. Indien dit niet zo is, moet er meer manueel werk verricht worden door de geneeskundige. Dit manueel werk is gevoelig voor medische fouten. Goede interoperabiliteit wordt alleen maar belangrijker nu dat mobiele toestellen hun intrede doen in de zorgverlening. Een voorbeeld hiervan is telemonitoring. Hierbij wordt een pati\"{e}nt op afstand geobserveerd. De pati\"{e}nt laat zijn of haar status weten aan de geneeskundige door metingen gemaakt door een toestel door te sturen naar de medische institutie.

Als oplossing wordt een modulair dashboard applicatie voorgesteld. Deze applicatie laat geneeskundingen toe om een dashboard samen te stellen naargelang de noden van elke pati\"{e}nt. Het dashboard voorziet modules die elk bepaalde functies aanbieden. Het beheren van voorschriften is hiervan een voorbeeld. Daarbij zijn alle modules gebruiksvriendelijk. Langs de kant van de ontwikkelaar is het simpel om een module te cre\"{e}ren en toe te voegen aan het dashboard. Dit moet goede interoperabiliteit toelaten.

Door een gebruiksvriendelijke applicatie te voorzien zouden er minder medische fouten gemaakt moeten worden. Het modulaire aspect helpt geneeskundigen om zich snel aan te kunnen passen aan veranderingen. Dit is mogelijk omdat elke verandering slechts \'{e}\'{e}n enkele module aanpast. De algehele werking van het dashboard zal dus niet veranderen.

Er is een prototype gemaakt om de gebruiksvriendelijkheid van het dashboard te toetsen. Nadat het prototype ontwikkeld was, is er een usability test opgesteld. Zes personen hebben aan de test meegedaan, wat zeer belangrijke feedback opleverde.

\section{Resultaten}

Het prototype werd door alle testpersonen als gebruiksvriendelijk ervaren. De testpersonen vertegenwoordigden verschillende rollen binnen de zorgverleningssector, zoals verpleegkunde en intensieve zorgverlening. De feedback resulteerde in meerdere verbeterpunten voor zowel de individuele modules als voor het gehele dashboard. Voor de toekomst is het belangrijk dat er met deze feedback wordt rekening gehouden. Ook moet het interoperabiliteit aspect getest nog worden door enkele ontwikkelaars, wat gedurende de periode van de studie niet is kunnen gebeuren. Om nog verder in de toekomst te kijken, moet de dashboard applicatie voor een langere periode getest worden in een daadwerkelijke medische omgeving.

\end{document}
