\section{Design} \label{design}

After the literature study, a design for the dashboard solution was made. What follows is a short explanation of the design process that was used. Based on the topics mentioned in section \ref{background}, a proposal was made to tackle some of the highlighted problems. Hereafter, personas and scenarios hint at the functionality and usage of this new system. Last, the components of the system, together with paper-mockups, are described.

    \subsection{MuiCSer} \label{2_muicser}
    % description of process
    This section describes the MuiCSer framework, made for user-centered software engineering processes in a multidisciplinary context \cite{muicser}, which was used to create the prototype. This framework focuses on optimizing the user experience during the entire software engineering cycle to ensure that the end-user's needs are fulfilled. By combining user-centered design and software engineering principles the user experience of the final product can be improved substantially of the final product.

        \subsubsection{Process}
        
        \begin{figure}[!t]
            \centering
            \includegraphics[width=0.8\textwidth]{chapters/3_design/muicser}
            \caption{MuiCSer process}\label{fig:muicser}
        \end{figure}

        The MuiCSer process is summarized in figure \ref{fig:muicser}. After each phase, the result is evaluated, verified and validated to ensure that the required functionality is present. The received feedback can, in turn, be used to reiterate over the previous phase. On the figure, this is denoted with the light arrows, while the dark one represents the overall process direction.

        \paragraph{New or legacy system} At the start of the process an existing system in need of improvement is either evaluated or a new one has to be designed. This requires an analysis of the tasks and needs of the user, as well as the objects and resources required to perform these tasks. Personas and scenarios are the resulting artifacts of this phase. First, personas describe the personalities of the potential end users including hobbies, skills and the environment they surround themselves in \cite{persona_scenario}. Its goal is to uncover behavior patterns which can be of use when designing a user interface. Second, a scenario is a story describing the use of a fictitious system from the persona's point of view \cite{persona_scenario}. It tries to sketch the usage of the system for which a design must be made.

        \paragraph{Structured interaction analysis} During this phase, the results of the analysis are used to create task models. These models specify concrete tasks and goals which can be dissected into specific actions or steps the user has to take. These artifacts lay the foundation for designing a user interface which supports these tasks and goals.

        \paragraph{Low fidelity prototyping} When the actions have been specified using the task models, low fidelity prototypes can be designed. Paper sketches and mockups are such examples and are ideal for visualizing the layout of the software its user interface. Without spending too much time and resources, presenting such prototypes can yield valuable feedback from the end-user or customer. However, there is no interaction present. Typically multiple versions of these prototypes will be created until the customer is satisfied after which high fidelity prototypes can be developed.

        \paragraph{High fidelity prototyping} Creating high fidelity prototypes requires a lot more effort compared to low fidelity prototypes, as they offer functionality closely resembling the final product. However, the feedback will be much more valuable as not only design, but also functionality is tested.

        \paragraph{Final user interface} When the latest iteration of the high fidelity prototype satisfies all user requirements, the final user interface can be created. It would be beneficial to reuse the code from the prototype in order to save time and resources. As a final step, the task models are checked against the interface to check if all required functionality is present.

    \subsection{Proposal}
    
    The proposal features three core concepts which will be combined to form one cohesive solution: a dashboard, telemonitoring of chronic diseases, and customization. As mentioned in the introduction, multiple applications and integrating them pose several difficulties: data is spread and stored multiple times. Interaction between these applications is not present and integrating them requires work from the IT staff from the institution. This becomes increasingly difficult as more applications are added. To remedy this problem, a dashboard solution can gather all relevant data into one place.

    A module-based approach towards the dashboard allows easy customization and integration. Each module will serve its own purpose by showing data for one particular health parameter. As mentioned in section \ref{2_telemonitoring}, parameters such as heart rate and blood pressure are often measured to monitor multiple chronic diseases. Integration is handled by the fact that each module is independent and does not require other modules to work properly. This also helps with customization as a clinician can choose which modules are relevant for the patient in question.

    The modules that will be developed are for use in chronic disease monitoring. However, modules can be developed for purposes more fit for in-patient care. Examples are real-time monitoring, documenting diagnoses, and viewing of lab results. Which modules will be implemented and what each one of them tries to achieve is described in section \ref{design_modules}. The implementation section will provide more insight on how this independence is achieved and how future modules can be easily integrated.

    This combination allows the clinician to customize the dashboard according to each patient's needs, facilitating effective care. Data is found rapidly, albeit summarized or detailed, and not hidden away behind multiple screens. future applications should be able to create modules that will fit into the dashboard, negating the use of multiple applications. The following section describes how the system will work and how the users interact with it.

    \subsection{Personas \& scenarios}

    Based on the proposal mentioned in the previous section, personas and scenarios have been created to get a better understanding of its users and how the system will work. Each of these tries to highlight the problems the users face and how the new system tries to solve them.

        % highlights customization and dashboard
        \subsubsection{Jake}

        \paragraph{Persona} Jake is a 25-year-old male who currently works as a nurse at the hospital in his city. He has been working for 4 years for the hospital and lives alone in his apartment. Still being a young adult, Jake grew up with technology. As such he experiences no difficulties when using new software on his computer or smartphone. His hobbies include music, playing the guitar, and video gaming.

        Currently, the workflow at the hospital is dated. A new health information system was introduced to summarize and gather all medical data in a singular space. Because this is a monolithic system, a lot of features that Jake does not need still clutter the screen. Navigating the system is a pain and customization is not present. Jake wishes to only see the features he uses most while hiding the features he does not need.
        
        \paragraph{Scenario} Jake starts his first day using the new system by reading the manual that is accompanied with it. He starts the application and for each patient he is presented with a few standard layouts to choose from, based on illnesses: cardiovascular diseases, respiratory diseases, diabetes and a few others.

        After selecting a standard layout, Jake is given the opportunity to customize the dashboard. The dashboard contains all the wanted functionality of the system, where each ``block'' represents a module that can be added or deleted. Jake enters the edit mode enabling him to move the blocks around and arrange the order. He deletes a few modules and now he wants to add other modules. Jake opens the modules window where he is presented with all the available modules. A module is added to the dashboard when Jake selects it.

        By examining the new module Jake quickly notices that on the dashboard a summary is given. But when Jake clicks on the module, the block expands to fill the whole screen where detailed information is given. Jake feels he is in control of the system and that it will improve his productivity.\\

        \noindent This scenario highlights the benefits of customization. Hiding unwanted components, while adding the useful ones allow for a clear dashboard to be displayed. The user is in control and is able to quickly view data of importance. It is also possible to view detailed data.
        
        % highlights customization
        \subsubsection{Dan}

        \paragraph{Persona} Dan is a general practitioner since he graduated from university. He is on the job for 21 years and he is the preferred doctor in his town. Throughout the years Dan has used a multitude of systems and he always tries new ones to improve his workflow. Because he has been a general practitioner for such a long time, he has 500 patients that visit him at least twice a year. Some patients, especially the elderly, visit as much as once a month.

        Most of Dan's patients visit for illnesses such as fever and a cold. To diagnose these illnesses no data is necessarily needed, just a description of what the patient feels will suffice most of the time. If Dan performs such a diagnosis, he promptly adds it to the information system and to the electronic health record of the patient.

        However, if a patient visits that has a lot of problems regarding his blood pressure, then Dan has to perform a more complex diagnosis using historical data. In the current system that Dan uses, it is very difficult to search for this data. But what bugs Dan the most is that he has to do it every time this patient visits.

        \paragraph{Scenario} Because of Dan's ongoing curiosity, he tries a new health information system which allows customization for each specific patient. Because the diagnoses of most patients can be very different from time to time, Dan creates a default module group which is displayed for each patient, unless that patient has a specific module configuration. The default module group includes past diagnoses, known allergies, patient information (blood type, last weight, height\ldots) and medication list.

        The first patient of the day describes what sounds like a fever. Dan confirms this and prescribes the patient some medicine. The diagnosis is added to the `past diagnoses' module and the prescribed medication to the `medication list' module. Dan did not need other health data to perform the diagnosis. Therefore, Dan does not change the configuration of this patient.

        The next patient, an elderly woman, came for her second visit of this month. Dan knows from the past that it will probably be a heart problem. Dan searches for a `heart' module and adds it to the configuration of the elderly woman. Now both the default module group and a heart rate module are present, which is unnecessary according to Dan. He removes some modules of the default module group. Now, the next time the elderly woman visits, that configuration will be loaded.

        One specific patient had broken his arm three times in less than a year. When the patient came for a routine visit, Dan immediately added a module to easily view x-ray photos and view them in a timeline.\\

        \noindent Again, the benefits of system customization for each patient are obvious. Initially, it will take some time to configure each dashboard for all of the patients, but the system will try to provide default layouts useful for certain types of care. Once the configuration is done, the time spent during a consultation will decrease.
        
        % highlights telemonitoring
        \subsubsection{Emily}

        \paragraph{Persona} Just like Dan, Emily has been a successful general practitioner for quite some time. However, she has different needs of an EHRS. Between each patient visit, there is a period of time in which Emily does not know what happens to patients regarding certain parameters. For example, if a person has to regularly measure his heart rate and blood pressure because of cardiac disease, it is imperative that the doctor is made aware of these values. If Emily sees that these values are not looking well, she can contact the patient to come in for a checkup.

        If a patient has sleep issues, Emily wishes to not see detailed measurement values, but regular descriptions of the night’s sleep. This includes the hours of sleep, amount of wake-ups, subjective feeling of tiredness. Currently, Emily has no way of regularly receiving this information without having the patient visit, because it needs to be documented at the practice.
        
        \paragraph{Scenario} Emily recently received notice of a new platform that includes telemonitoring support. Several mobile applications are developed that can send data using an API to the platform, which in turn processes the values and can notify Emily of any anomalies. It includes customization for certain parameters in which Emily can individually assign thresholds for each patient.

        A patient who recently had a cardiac arrest is continuing rehabilitation at home. However, the patient has chest pains and pays Emily a visit. She tells the patient to regularly measure his blood pressure and heart rate, and to take note of these values in a mobile application. This application also allows taking general notes, such as feeling pain or having caught a cold. After they have scheduled the next visit, the patient is sent home.
        
        As the next few weeks pass by, Emily is notified that this patient has crossed a threshold regarding his blood pressure. Immediately Emily checks the measured data and sees a graph of all measured values. This dataset delivers insight into the history of the patient and Emily sees that there is currently no need to panic. She decides to not take action and configures a weekly reminder to keep monitoring the blood pressure.
        
        The next week, Emily receives a notification that the patient has made a note. It reads that he experienced chest pains. Again, Emily takes a look at the blood pressure data and sees a worsening trend leading to hypertension. Emily decides to call the patient to schedule an early visit. The system helped Emily to intervene as soon as the situation seemed to worsen while avoiding having the patient visit too early which in turn saves Emily time.
        
        Another patient has sleep issues. Emily encourages the patient to use a sleep monitor, which is a wristband. This device is connected to a smartphone which communicates the quantitative data to the new platform. The application on the smartphone allows the patient to take note of qualitative data such as a general description of the night or what food he/she ate. One night the patient slept only three hours and took note that he went out and drank a lot of alcohol. This could explain for example the bad sleeping rhythm for the next few days. Emily wishes to keep track of this patient on a weekly basis and configures the platform to notify her.\\

        \noindent Here, the effects of telemonitoring are highlighted. By generating reports and alerts, the caregiver can intervene when necessary. Combine this with the aforementioned dashboard platform, greater efficiency of care can be accomplished.

        % highlights integration
        \subsubsection{Anna}

        \paragraph{Persona} Anna is a software engineer working at the local hospital. From the early 2000's, she was responsible for the integration of health software systems. As such, she has experienced the continuous change associated with health software. Each unit of the hospital uses different software, while they all share a general system to view patient records. The interaction between these applications is one of Anna's responsibilities.

        As time goes on, new software and medical instruments that improve workflow are purchased. Sometimes these replace old systems, otherwise they are added to work alongside them. Both are becoming increasingly difficult to do, as the codebase of the EHRS grows. When an older system gets replaced, Anna has to weed through the code and delete little pieces, hoping other parts wont break. It is a frustrating task as often band-aid solutions are applied.

        \paragraph{Scenario} A new EHRS was installed that should improve integration of other applications. Anna skims through the documentation and starts the transition process. She quickly notices that for each device or application she needs to create a backend data structure and a module that represents it in the EHRS. The new EHRS is designed that these modules can serve as complete standalone applications or be part of a set of modules.

        After the initial transition period, everything is put into place. Not soon after, a new bedside monitors with accompanying software was purchased. These replace the old bedside monitors and consequently, the software. Thanks to the new EHRS, Anna deletes the old module without much hassle. Also, she refit the backend data structure to work with the new device's readings. This ensures that the old device data is still intact. Anna designed the new module to be similar in appearance and functionality. After implementing the module, the staff barely notices any change and workflow resumes with minor disruption.\\

        \noindent This persona and scenario highlight the importance of integration. Significant time and effort can be saved if this process was simplified. Also, maintenance is easier due to looser coupling of the modules.

    \subsection{Modules} \label{design_modules}
    
    As mentioned in the proposal, the dashboard will feature a module-based design. This section will describe the modules that will be built in the prototype. For each module, the reasons behind the design choices are given, as well as a low-fidelity paper mockup. Keep in mind that all modules are primarily chosen to aid with chronic disease management.

    Each module has its own design, which is in line with the principles mentioned in section \ref{2_dashboards}. For each module we ask the following questions:
    \begin{itemize}
        \item What functionality does the module offer?
        \item What data will it represent?
        \item How do we show as much data as possible at a glance?
        \item How are summaries generated?
        \item How are the detailed values shown?
    \end{itemize}

    Some functionality is present in all modules. These include resizing the module to a larger or smaller version with varying amounts of detail, increasing the customization options. The smaller the module, the more data is aggregated to a summary. Each module can be freely placed anywhere on the dashboard. Also, all modules can navigate through historical data either by week or by month. The ability to set thresholds for each patient is always present, which is necessary to generate appropriate alerts.

    Based on the telemonitoring literature review from section \ref{2_telemonitoring}, the following modules will be created:
    \begin{itemize}
        \item Heart rate: cardiovascular disease, diabetes, and COPD.
        \item Blood pressure: cardiovascular disease and diabetes.
        \item Blood sugar: diabetes.
        \item Weight: cardiovascular disease and diabetes.
        \item Oxygen saturation: cardiovascular disease and COPD.
        \item Medication: cardiovascular disease and diabetes.
    \end{itemize}

        \subsubsection{Heart rate}


        \subsubsection{Blood pressure}


        \subsubsection{Blood sugar}


        \subsubsection{Weight}

        
        \subsubsection{Oxygen saturation}


        \subsubsection{Medication}


    % dashboard: saving layouts, adding modules...
    \subsection{Other components}

