\section{Introduction}

Due to the continuous growth of ubiquitous computing, the possibility of continuing rehabilitation at home rises. The caregiver monitors the patient at a distance and if needed, the patient is contacted. As a result, resources on both the caregiver and patient side can be saved. This process is called telemonitoring. Research primarily investigated telemonitoring for chronic disease management. Because of the enduring nature of these conditions, patients report the status of their condition using a device such as a smartphone and a digital blood pressure monitor.

\sloppy The referred telemonitoring literature always developed new platforms. There was no mention of integration or actual adoption of these platforms. Using different applications to monitor, for example, diabetes and cardiovascular diseases would cause redundancy as parameters such as blood pressure are important for both disease groups. These values need to be updated at two locations, unless they share a common backend system. However, this is in research applications never the case. Data duplication raises several issues: more storage space, data needs more maintenance, and integration becomes increasingly difficult. Therefore, it can be beneficial if all health parameter data is stored and monitored at one single location.

Most health care institutions have transitioned to electronic patient records. Software solutions interact with these records to automatically generate statistics, decision support, and many types of alerts. Monolithic software applications are often used for more general care while specific software applies to specific types of care. There are benefits and drawbacks to both, but no literature was found which reported on integrating the specific applications with a monolithic system. In a perfect world, one singular system provides all the required functionality. Again, integration is key here.

Another gap in health software research is the effect of customization. Each patient is different and a customized rehabilitation trajectory is preferred. Current health systems provide a general view for all patients where relevant information is buried behind many screens and menus. Should the software allow clinicians to indicate what is of importance, this information can be reached much faster. For example, a dashboard-like screen is generated which gathers all important data as indicated by the caregiver at one location. This dashboard can generate summaries while details are one click away. 

The issues mentioned in the last few paragraphs signify the need to see \emph{relevant} information at a glance. Not only for telemonitoring, but also for general care given in a hospital setting, this can be beneficial. Together with customization, care is delivered with greater efficiency. This thesis proposes a dashboard design that allows the caregiver to customize its functionality according to the patient's needs in terms of treatment. Its components will mainly focus on chronic disease management for use in telemonitoring, due to the amount literature that exists for this topic. 

The architecture of the dashboard prototype tries to facilitate easy integration. It should be simple to add, update, and remove functional components without changing the system at its core. After reading this thesis, an external developer should be able to create his own components. 

First, to get a thorough understanding of related subjects, background information is given in section \ref{background}. The discussed topics include electronic health record systems, chronic disease management, a literature review on telemonitoring, dashboard design, and health data privacy and standards. Based on this background information a dashboard design is proposed (section \ref{design}). This section highlights the design process and the made choices. Also, personas, scenarios, and paper mockups give a first indication of how the system will be used. Implementation of the dashboard started after the design process. In section \ref{implementation} the used frameworks and changes compared to the paper mockups are described. The realization of customizing the dashboard and easy integration is also covered in this section.

As soon as the implementation was finished, a usability test was conducted. Section \ref{usabilitytest} describes the research questions, the test setup, the created documents, and information concerning the testers. The results of the test are discussed in the following section. Finally, the thesis concludes with a summary of its findings and steps for future work.