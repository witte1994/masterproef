\section{Introduction}

Due to the continuous growth of ubiquitous computing, the possibility of continuing rehabilitation at home rises. The caregiver monitors the patient at a distance and if needed, the patient is contacted. As a result, resources on both the caregiver and patient side can be saved. This process is called telemonitoring. Research primarily investigated telemonitoring for chronic disease management. Because of the enduring nature of these conditions, patients report the status of their condition using a device such as a smartphone and a digital blood pressure monitor.

The consulted telemonitoring studies resulted in the development of many software platforms, with each trying to monitor patients at home as effective as possible. However, there was never mention of using or expanding an existing platform, only of existing measuring devices. Using different applications to monitor, for example, diabetes and cardiovascular diseases would cause redundancy as parameters such as blood pressure are important for both disease groups. These values need to be updated at two locations, unless they share a common backend system. However, in research applications this is never the case. Data duplication raises several issues: more storage space, data needs more maintenance, and integration becomes increasingly difficult. If a certain workflow requires many applications, time is already lost by switching between them. Therefore, it can be beneficial if all health data is stored and monitored at a single location.

Most health care institutions have transitioned to electronic patient records. Software solutions interact with these records to automatically generate statistics, decision support, and many types of alerts. Monolithic software applications are often used for more general care while specific software applies to specific types of care. There are benefits and drawbacks to both, but no literature was found which reported on the integration of the specific applications with a monolithic system. In a perfect world, a single system provides all the required functionality, as was implied in the previous paragraph.

Another gap in health software research is the effect of customization. Each patient is different and a customized rehabilitation trajectory is preferred. Current health systems provide a general view for all patients where information is buried behind many screens and menus. For example, a patient who suffers from cardiovascular disease visits his general practitioner on a regular basis. Each visit the practitioner has to navigate to the data relevant to the disease. Should the software allow clinicians to indicate what is of importance, relevant information can be reached much faster. For example, a dashboard-like screen is generated which gathers all important data as indicated by the caregiver at one location. This dashboard can, for example, generate summaries while details are one click away. 

The main research topic of this thesis is the effect of customization and system integration on health software. One can speculate that the presence these two components increases the efficiency of care delivery. On one hand, clinicians save time by having easy access to their necessary tools. On the other hand, system administrators can easily expand a central software platform with, for example, 3rd party components. Also, maintenance should be easier as all health data is centralized. By saving man-hours, the expenditure of health institutions reduces. For patients, less time is spent at visits, which in turn lowers the costs.

This thesis proposes a dashboard design that allows the caregiver to customize its functionality according to the patient's needs in terms of treatment. Its components will mainly focus on chronic disease management for use in telemonitoring, due to customization possibilities and the amount literature that exists on this topic. The architecture of the dashboard prototype tries to facilitate easy integration. It should be simple to add, update, and remove functional components without changing the system at its core. After reading this thesis, an external developer should be able to create his own components and plug them into the platform. 

First, to get a thorough understanding of related subjects, background information is given in section \ref{background}. The discussed topics include electronic health record systems, chronic disease management, a literature review on telemonitoring, dashboard design, and health data privacy and standards. Based on this background information a dashboard design is proposed (section \ref{design}). This section highlights the design process and the made choices. Also, personas, scenarios, and paper mockups give a first indication of how the system will be used. Implementation of the dashboard started after the design process. In section \ref{implementation} the used frameworks and changes compared to the paper mockups are described. The realization of customization and integration is also covered in this section. As soon as the implementation was finished, a usability test was conducted. Section \ref{usabilitytest} describes the research questions, the test setup, the created documents, and information concerning the testers. The results of the test are discussed in the following section. Finally, the thesis concludes with a summary of its findings and steps for future work.